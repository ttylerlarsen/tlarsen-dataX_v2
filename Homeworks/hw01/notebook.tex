
% Default to the notebook output style

    


% Inherit from the specified cell style.




    
\documentclass[11pt]{article}

    
    
    \usepackage[T1]{fontenc}
    % Nicer default font (+ math font) than Computer Modern for most use cases
    \usepackage{mathpazo}

    % Basic figure setup, for now with no caption control since it's done
    % automatically by Pandoc (which extracts ![](path) syntax from Markdown).
    \usepackage{graphicx}
    % We will generate all images so they have a width \maxwidth. This means
    % that they will get their normal width if they fit onto the page, but
    % are scaled down if they would overflow the margins.
    \makeatletter
    \def\maxwidth{\ifdim\Gin@nat@width>\linewidth\linewidth
    \else\Gin@nat@width\fi}
    \makeatother
    \let\Oldincludegraphics\includegraphics
    % Set max figure width to be 80% of text width, for now hardcoded.
    \renewcommand{\includegraphics}[1]{\Oldincludegraphics[width=.8\maxwidth]{#1}}
    % Ensure that by default, figures have no caption (until we provide a
    % proper Figure object with a Caption API and a way to capture that
    % in the conversion process - todo).
    \usepackage{caption}
    \DeclareCaptionLabelFormat{nolabel}{}
    \captionsetup{labelformat=nolabel}

    \usepackage{adjustbox} % Used to constrain images to a maximum size 
    \usepackage{xcolor} % Allow colors to be defined
    \usepackage{enumerate} % Needed for markdown enumerations to work
    \usepackage{geometry} % Used to adjust the document margins
    \usepackage{amsmath} % Equations
    \usepackage{amssymb} % Equations
    \usepackage{textcomp} % defines textquotesingle
    % Hack from http://tex.stackexchange.com/a/47451/13684:
    \AtBeginDocument{%
        \def\PYZsq{\textquotesingle}% Upright quotes in Pygmentized code
    }
    \usepackage{upquote} % Upright quotes for verbatim code
    \usepackage{eurosym} % defines \euro
    \usepackage[mathletters]{ucs} % Extended unicode (utf-8) support
    \usepackage[utf8x]{inputenc} % Allow utf-8 characters in the tex document
    \usepackage{fancyvrb} % verbatim replacement that allows latex
    \usepackage{grffile} % extends the file name processing of package graphics 
                         % to support a larger range 
    % The hyperref package gives us a pdf with properly built
    % internal navigation ('pdf bookmarks' for the table of contents,
    % internal cross-reference links, web links for URLs, etc.)
    \usepackage{hyperref}
    \usepackage{longtable} % longtable support required by pandoc >1.10
    \usepackage{booktabs}  % table support for pandoc > 1.12.2
    \usepackage[inline]{enumitem} % IRkernel/repr support (it uses the enumerate* environment)
    \usepackage[normalem]{ulem} % ulem is needed to support strikethroughs (\sout)
                                % normalem makes italics be italics, not underlines
    

    
    
    % Colors for the hyperref package
    \definecolor{urlcolor}{rgb}{0,.145,.698}
    \definecolor{linkcolor}{rgb}{.71,0.21,0.01}
    \definecolor{citecolor}{rgb}{.12,.54,.11}

    % ANSI colors
    \definecolor{ansi-black}{HTML}{3E424D}
    \definecolor{ansi-black-intense}{HTML}{282C36}
    \definecolor{ansi-red}{HTML}{E75C58}
    \definecolor{ansi-red-intense}{HTML}{B22B31}
    \definecolor{ansi-green}{HTML}{00A250}
    \definecolor{ansi-green-intense}{HTML}{007427}
    \definecolor{ansi-yellow}{HTML}{DDB62B}
    \definecolor{ansi-yellow-intense}{HTML}{B27D12}
    \definecolor{ansi-blue}{HTML}{208FFB}
    \definecolor{ansi-blue-intense}{HTML}{0065CA}
    \definecolor{ansi-magenta}{HTML}{D160C4}
    \definecolor{ansi-magenta-intense}{HTML}{A03196}
    \definecolor{ansi-cyan}{HTML}{60C6C8}
    \definecolor{ansi-cyan-intense}{HTML}{258F8F}
    \definecolor{ansi-white}{HTML}{C5C1B4}
    \definecolor{ansi-white-intense}{HTML}{A1A6B2}

    % commands and environments needed by pandoc snippets
    % extracted from the output of `pandoc -s`
    \providecommand{\tightlist}{%
      \setlength{\itemsep}{0pt}\setlength{\parskip}{0pt}}
    \DefineVerbatimEnvironment{Highlighting}{Verbatim}{commandchars=\\\{\}}
    % Add ',fontsize=\small' for more characters per line
    \newenvironment{Shaded}{}{}
    \newcommand{\KeywordTok}[1]{\textcolor[rgb]{0.00,0.44,0.13}{\textbf{{#1}}}}
    \newcommand{\DataTypeTok}[1]{\textcolor[rgb]{0.56,0.13,0.00}{{#1}}}
    \newcommand{\DecValTok}[1]{\textcolor[rgb]{0.25,0.63,0.44}{{#1}}}
    \newcommand{\BaseNTok}[1]{\textcolor[rgb]{0.25,0.63,0.44}{{#1}}}
    \newcommand{\FloatTok}[1]{\textcolor[rgb]{0.25,0.63,0.44}{{#1}}}
    \newcommand{\CharTok}[1]{\textcolor[rgb]{0.25,0.44,0.63}{{#1}}}
    \newcommand{\StringTok}[1]{\textcolor[rgb]{0.25,0.44,0.63}{{#1}}}
    \newcommand{\CommentTok}[1]{\textcolor[rgb]{0.38,0.63,0.69}{\textit{{#1}}}}
    \newcommand{\OtherTok}[1]{\textcolor[rgb]{0.00,0.44,0.13}{{#1}}}
    \newcommand{\AlertTok}[1]{\textcolor[rgb]{1.00,0.00,0.00}{\textbf{{#1}}}}
    \newcommand{\FunctionTok}[1]{\textcolor[rgb]{0.02,0.16,0.49}{{#1}}}
    \newcommand{\RegionMarkerTok}[1]{{#1}}
    \newcommand{\ErrorTok}[1]{\textcolor[rgb]{1.00,0.00,0.00}{\textbf{{#1}}}}
    \newcommand{\NormalTok}[1]{{#1}}
    
    % Additional commands for more recent versions of Pandoc
    \newcommand{\ConstantTok}[1]{\textcolor[rgb]{0.53,0.00,0.00}{{#1}}}
    \newcommand{\SpecialCharTok}[1]{\textcolor[rgb]{0.25,0.44,0.63}{{#1}}}
    \newcommand{\VerbatimStringTok}[1]{\textcolor[rgb]{0.25,0.44,0.63}{{#1}}}
    \newcommand{\SpecialStringTok}[1]{\textcolor[rgb]{0.73,0.40,0.53}{{#1}}}
    \newcommand{\ImportTok}[1]{{#1}}
    \newcommand{\DocumentationTok}[1]{\textcolor[rgb]{0.73,0.13,0.13}{\textit{{#1}}}}
    \newcommand{\AnnotationTok}[1]{\textcolor[rgb]{0.38,0.63,0.69}{\textbf{\textit{{#1}}}}}
    \newcommand{\CommentVarTok}[1]{\textcolor[rgb]{0.38,0.63,0.69}{\textbf{\textit{{#1}}}}}
    \newcommand{\VariableTok}[1]{\textcolor[rgb]{0.10,0.09,0.49}{{#1}}}
    \newcommand{\ControlFlowTok}[1]{\textcolor[rgb]{0.00,0.44,0.13}{\textbf{{#1}}}}
    \newcommand{\OperatorTok}[1]{\textcolor[rgb]{0.40,0.40,0.40}{{#1}}}
    \newcommand{\BuiltInTok}[1]{{#1}}
    \newcommand{\ExtensionTok}[1]{{#1}}
    \newcommand{\PreprocessorTok}[1]{\textcolor[rgb]{0.74,0.48,0.00}{{#1}}}
    \newcommand{\AttributeTok}[1]{\textcolor[rgb]{0.49,0.56,0.16}{{#1}}}
    \newcommand{\InformationTok}[1]{\textcolor[rgb]{0.38,0.63,0.69}{\textbf{\textit{{#1}}}}}
    \newcommand{\WarningTok}[1]{\textcolor[rgb]{0.38,0.63,0.69}{\textbf{\textit{{#1}}}}}
    
    
    % Define a nice break command that doesn't care if a line doesn't already
    % exist.
    \def\br{\hspace*{\fill} \\* }
    % Math Jax compatability definitions
    \def\gt{>}
    \def\lt{<}
    % Document parameters
    \title{Data-X HW1 Sp18}
    
    
    

    % Pygments definitions
    
\makeatletter
\def\PY@reset{\let\PY@it=\relax \let\PY@bf=\relax%
    \let\PY@ul=\relax \let\PY@tc=\relax%
    \let\PY@bc=\relax \let\PY@ff=\relax}
\def\PY@tok#1{\csname PY@tok@#1\endcsname}
\def\PY@toks#1+{\ifx\relax#1\empty\else%
    \PY@tok{#1}\expandafter\PY@toks\fi}
\def\PY@do#1{\PY@bc{\PY@tc{\PY@ul{%
    \PY@it{\PY@bf{\PY@ff{#1}}}}}}}
\def\PY#1#2{\PY@reset\PY@toks#1+\relax+\PY@do{#2}}

\expandafter\def\csname PY@tok@w\endcsname{\def\PY@tc##1{\textcolor[rgb]{0.73,0.73,0.73}{##1}}}
\expandafter\def\csname PY@tok@c\endcsname{\let\PY@it=\textit\def\PY@tc##1{\textcolor[rgb]{0.25,0.50,0.50}{##1}}}
\expandafter\def\csname PY@tok@cp\endcsname{\def\PY@tc##1{\textcolor[rgb]{0.74,0.48,0.00}{##1}}}
\expandafter\def\csname PY@tok@k\endcsname{\let\PY@bf=\textbf\def\PY@tc##1{\textcolor[rgb]{0.00,0.50,0.00}{##1}}}
\expandafter\def\csname PY@tok@kp\endcsname{\def\PY@tc##1{\textcolor[rgb]{0.00,0.50,0.00}{##1}}}
\expandafter\def\csname PY@tok@kt\endcsname{\def\PY@tc##1{\textcolor[rgb]{0.69,0.00,0.25}{##1}}}
\expandafter\def\csname PY@tok@o\endcsname{\def\PY@tc##1{\textcolor[rgb]{0.40,0.40,0.40}{##1}}}
\expandafter\def\csname PY@tok@ow\endcsname{\let\PY@bf=\textbf\def\PY@tc##1{\textcolor[rgb]{0.67,0.13,1.00}{##1}}}
\expandafter\def\csname PY@tok@nb\endcsname{\def\PY@tc##1{\textcolor[rgb]{0.00,0.50,0.00}{##1}}}
\expandafter\def\csname PY@tok@nf\endcsname{\def\PY@tc##1{\textcolor[rgb]{0.00,0.00,1.00}{##1}}}
\expandafter\def\csname PY@tok@nc\endcsname{\let\PY@bf=\textbf\def\PY@tc##1{\textcolor[rgb]{0.00,0.00,1.00}{##1}}}
\expandafter\def\csname PY@tok@nn\endcsname{\let\PY@bf=\textbf\def\PY@tc##1{\textcolor[rgb]{0.00,0.00,1.00}{##1}}}
\expandafter\def\csname PY@tok@ne\endcsname{\let\PY@bf=\textbf\def\PY@tc##1{\textcolor[rgb]{0.82,0.25,0.23}{##1}}}
\expandafter\def\csname PY@tok@nv\endcsname{\def\PY@tc##1{\textcolor[rgb]{0.10,0.09,0.49}{##1}}}
\expandafter\def\csname PY@tok@no\endcsname{\def\PY@tc##1{\textcolor[rgb]{0.53,0.00,0.00}{##1}}}
\expandafter\def\csname PY@tok@nl\endcsname{\def\PY@tc##1{\textcolor[rgb]{0.63,0.63,0.00}{##1}}}
\expandafter\def\csname PY@tok@ni\endcsname{\let\PY@bf=\textbf\def\PY@tc##1{\textcolor[rgb]{0.60,0.60,0.60}{##1}}}
\expandafter\def\csname PY@tok@na\endcsname{\def\PY@tc##1{\textcolor[rgb]{0.49,0.56,0.16}{##1}}}
\expandafter\def\csname PY@tok@nt\endcsname{\let\PY@bf=\textbf\def\PY@tc##1{\textcolor[rgb]{0.00,0.50,0.00}{##1}}}
\expandafter\def\csname PY@tok@nd\endcsname{\def\PY@tc##1{\textcolor[rgb]{0.67,0.13,1.00}{##1}}}
\expandafter\def\csname PY@tok@s\endcsname{\def\PY@tc##1{\textcolor[rgb]{0.73,0.13,0.13}{##1}}}
\expandafter\def\csname PY@tok@sd\endcsname{\let\PY@it=\textit\def\PY@tc##1{\textcolor[rgb]{0.73,0.13,0.13}{##1}}}
\expandafter\def\csname PY@tok@si\endcsname{\let\PY@bf=\textbf\def\PY@tc##1{\textcolor[rgb]{0.73,0.40,0.53}{##1}}}
\expandafter\def\csname PY@tok@se\endcsname{\let\PY@bf=\textbf\def\PY@tc##1{\textcolor[rgb]{0.73,0.40,0.13}{##1}}}
\expandafter\def\csname PY@tok@sr\endcsname{\def\PY@tc##1{\textcolor[rgb]{0.73,0.40,0.53}{##1}}}
\expandafter\def\csname PY@tok@ss\endcsname{\def\PY@tc##1{\textcolor[rgb]{0.10,0.09,0.49}{##1}}}
\expandafter\def\csname PY@tok@sx\endcsname{\def\PY@tc##1{\textcolor[rgb]{0.00,0.50,0.00}{##1}}}
\expandafter\def\csname PY@tok@m\endcsname{\def\PY@tc##1{\textcolor[rgb]{0.40,0.40,0.40}{##1}}}
\expandafter\def\csname PY@tok@gh\endcsname{\let\PY@bf=\textbf\def\PY@tc##1{\textcolor[rgb]{0.00,0.00,0.50}{##1}}}
\expandafter\def\csname PY@tok@gu\endcsname{\let\PY@bf=\textbf\def\PY@tc##1{\textcolor[rgb]{0.50,0.00,0.50}{##1}}}
\expandafter\def\csname PY@tok@gd\endcsname{\def\PY@tc##1{\textcolor[rgb]{0.63,0.00,0.00}{##1}}}
\expandafter\def\csname PY@tok@gi\endcsname{\def\PY@tc##1{\textcolor[rgb]{0.00,0.63,0.00}{##1}}}
\expandafter\def\csname PY@tok@gr\endcsname{\def\PY@tc##1{\textcolor[rgb]{1.00,0.00,0.00}{##1}}}
\expandafter\def\csname PY@tok@ge\endcsname{\let\PY@it=\textit}
\expandafter\def\csname PY@tok@gs\endcsname{\let\PY@bf=\textbf}
\expandafter\def\csname PY@tok@gp\endcsname{\let\PY@bf=\textbf\def\PY@tc##1{\textcolor[rgb]{0.00,0.00,0.50}{##1}}}
\expandafter\def\csname PY@tok@go\endcsname{\def\PY@tc##1{\textcolor[rgb]{0.53,0.53,0.53}{##1}}}
\expandafter\def\csname PY@tok@gt\endcsname{\def\PY@tc##1{\textcolor[rgb]{0.00,0.27,0.87}{##1}}}
\expandafter\def\csname PY@tok@err\endcsname{\def\PY@bc##1{\setlength{\fboxsep}{0pt}\fcolorbox[rgb]{1.00,0.00,0.00}{1,1,1}{\strut ##1}}}
\expandafter\def\csname PY@tok@kc\endcsname{\let\PY@bf=\textbf\def\PY@tc##1{\textcolor[rgb]{0.00,0.50,0.00}{##1}}}
\expandafter\def\csname PY@tok@kd\endcsname{\let\PY@bf=\textbf\def\PY@tc##1{\textcolor[rgb]{0.00,0.50,0.00}{##1}}}
\expandafter\def\csname PY@tok@kn\endcsname{\let\PY@bf=\textbf\def\PY@tc##1{\textcolor[rgb]{0.00,0.50,0.00}{##1}}}
\expandafter\def\csname PY@tok@kr\endcsname{\let\PY@bf=\textbf\def\PY@tc##1{\textcolor[rgb]{0.00,0.50,0.00}{##1}}}
\expandafter\def\csname PY@tok@bp\endcsname{\def\PY@tc##1{\textcolor[rgb]{0.00,0.50,0.00}{##1}}}
\expandafter\def\csname PY@tok@fm\endcsname{\def\PY@tc##1{\textcolor[rgb]{0.00,0.00,1.00}{##1}}}
\expandafter\def\csname PY@tok@vc\endcsname{\def\PY@tc##1{\textcolor[rgb]{0.10,0.09,0.49}{##1}}}
\expandafter\def\csname PY@tok@vg\endcsname{\def\PY@tc##1{\textcolor[rgb]{0.10,0.09,0.49}{##1}}}
\expandafter\def\csname PY@tok@vi\endcsname{\def\PY@tc##1{\textcolor[rgb]{0.10,0.09,0.49}{##1}}}
\expandafter\def\csname PY@tok@vm\endcsname{\def\PY@tc##1{\textcolor[rgb]{0.10,0.09,0.49}{##1}}}
\expandafter\def\csname PY@tok@sa\endcsname{\def\PY@tc##1{\textcolor[rgb]{0.73,0.13,0.13}{##1}}}
\expandafter\def\csname PY@tok@sb\endcsname{\def\PY@tc##1{\textcolor[rgb]{0.73,0.13,0.13}{##1}}}
\expandafter\def\csname PY@tok@sc\endcsname{\def\PY@tc##1{\textcolor[rgb]{0.73,0.13,0.13}{##1}}}
\expandafter\def\csname PY@tok@dl\endcsname{\def\PY@tc##1{\textcolor[rgb]{0.73,0.13,0.13}{##1}}}
\expandafter\def\csname PY@tok@s2\endcsname{\def\PY@tc##1{\textcolor[rgb]{0.73,0.13,0.13}{##1}}}
\expandafter\def\csname PY@tok@sh\endcsname{\def\PY@tc##1{\textcolor[rgb]{0.73,0.13,0.13}{##1}}}
\expandafter\def\csname PY@tok@s1\endcsname{\def\PY@tc##1{\textcolor[rgb]{0.73,0.13,0.13}{##1}}}
\expandafter\def\csname PY@tok@mb\endcsname{\def\PY@tc##1{\textcolor[rgb]{0.40,0.40,0.40}{##1}}}
\expandafter\def\csname PY@tok@mf\endcsname{\def\PY@tc##1{\textcolor[rgb]{0.40,0.40,0.40}{##1}}}
\expandafter\def\csname PY@tok@mh\endcsname{\def\PY@tc##1{\textcolor[rgb]{0.40,0.40,0.40}{##1}}}
\expandafter\def\csname PY@tok@mi\endcsname{\def\PY@tc##1{\textcolor[rgb]{0.40,0.40,0.40}{##1}}}
\expandafter\def\csname PY@tok@il\endcsname{\def\PY@tc##1{\textcolor[rgb]{0.40,0.40,0.40}{##1}}}
\expandafter\def\csname PY@tok@mo\endcsname{\def\PY@tc##1{\textcolor[rgb]{0.40,0.40,0.40}{##1}}}
\expandafter\def\csname PY@tok@ch\endcsname{\let\PY@it=\textit\def\PY@tc##1{\textcolor[rgb]{0.25,0.50,0.50}{##1}}}
\expandafter\def\csname PY@tok@cm\endcsname{\let\PY@it=\textit\def\PY@tc##1{\textcolor[rgb]{0.25,0.50,0.50}{##1}}}
\expandafter\def\csname PY@tok@cpf\endcsname{\let\PY@it=\textit\def\PY@tc##1{\textcolor[rgb]{0.25,0.50,0.50}{##1}}}
\expandafter\def\csname PY@tok@c1\endcsname{\let\PY@it=\textit\def\PY@tc##1{\textcolor[rgb]{0.25,0.50,0.50}{##1}}}
\expandafter\def\csname PY@tok@cs\endcsname{\let\PY@it=\textit\def\PY@tc##1{\textcolor[rgb]{0.25,0.50,0.50}{##1}}}

\def\PYZbs{\char`\\}
\def\PYZus{\char`\_}
\def\PYZob{\char`\{}
\def\PYZcb{\char`\}}
\def\PYZca{\char`\^}
\def\PYZam{\char`\&}
\def\PYZlt{\char`\<}
\def\PYZgt{\char`\>}
\def\PYZsh{\char`\#}
\def\PYZpc{\char`\%}
\def\PYZdl{\char`\$}
\def\PYZhy{\char`\-}
\def\PYZsq{\char`\'}
\def\PYZdq{\char`\"}
\def\PYZti{\char`\~}
% for compatibility with earlier versions
\def\PYZat{@}
\def\PYZlb{[}
\def\PYZrb{]}
\makeatother


    % Exact colors from NB
    \definecolor{incolor}{rgb}{0.0, 0.0, 0.5}
    \definecolor{outcolor}{rgb}{0.545, 0.0, 0.0}



    
    % Prevent overflowing lines due to hard-to-break entities
    \sloppy 
    % Setup hyperref package
    \hypersetup{
      breaklinks=true,  % so long urls are correctly broken across lines
      colorlinks=true,
      urlcolor=urlcolor,
      linkcolor=linkcolor,
      citecolor=citecolor,
      }
    % Slightly bigger margins than the latex defaults
    
    \geometry{verbose,tmargin=1in,bmargin=1in,lmargin=1in,rmargin=1in}
    
    

    \begin{document}
    
    
    \maketitle
    
    

    
    \hypertarget{homework-1}{%
\section{Homework 1}\label{homework-1}}

    In this homework, you will get a chance to do some exercises with Numpy,
Pandas, and Matplotlib to show us your understanding with this
libraries.

If you have questions, Google! Additionally you can ask your peers
questions on Piazza and/or go to Office Hours.

This homework is due \textbf{Thursday Feb.~8th, 2018 at 11:59 PM}.
Please upload your .ipynb to your private repo on Github. Additionally,
submit a pdf on bCourses and in the comment section include a link to
your private repo.

This homework is long, please start early!

    \begin{Verbatim}[commandchars=\\\{\}]
{\color{incolor}In [{\color{incolor}43}]:} \PY{k+kn}{import} \PY{n+nn}{math}
         \PY{k+kn}{import} \PY{n+nn}{numpy} \PY{k}{as} \PY{n+nn}{np}
         \PY{k+kn}{import} \PY{n+nn}{pandas} \PY{k}{as} \PY{n+nn}{pd}
         \PY{k+kn}{import} \PY{n+nn}{matplotlib}\PY{n+nn}{.}\PY{n+nn}{pyplot} \PY{k}{as} \PY{n+nn}{plt}
         \PY{o}{\PYZpc{}}\PY{k}{matplotlib} inline
\end{Verbatim}


    \begin{Verbatim}[commandchars=\\\{\}]

        ---------------------------------------------------------------------------

        ImportError                               Traceback (most recent call last)

        <ipython-input-43-105e9e7a0fde> in <module>()
    ----> 1 from math import mul
          2 import numpy as np
          3 import pandas as pd
          4 import matplotlib.pyplot as plt
          5 get\_ipython().run\_line\_magic('matplotlib', 'inline')


        ImportError: cannot import name 'mul'

    \end{Verbatim}

    \hypertarget{numpy-basics}{%
\subsection{NumPy Basics}\label{numpy-basics}}

    Create two numpy arrays (\texttt{a} and \texttt{b}). \texttt{a} should
be all integers between 10-19 (inclusive), and \texttt{b} should be ten
evenly spaced numbers between 1-7. Print the results below.

    \begin{Verbatim}[commandchars=\\\{\}]
{\color{incolor}In [{\color{incolor}66}]:} \PY{n}{a} \PY{o}{=} \PY{n}{np}\PY{o}{.}\PY{n}{arange}\PY{p}{(}\PY{l+m+mi}{10}\PY{p}{,}\PY{l+m+mi}{19}\PY{p}{)}
         \PY{n}{b} \PY{o}{=} \PY{n}{np}\PY{o}{.}\PY{n}{linspace}\PY{p}{(}\PY{l+m+mi}{1}\PY{p}{,}\PY{l+m+mi}{7}\PY{p}{,}\PY{l+m+mi}{10}\PY{p}{)}
         
         \PY{n+nb}{print}\PY{p}{(}\PY{l+s+s2}{\PYZdq{}}\PY{l+s+s2}{a: }\PY{l+s+s2}{\PYZdq{}}\PY{p}{,} \PY{n}{a}\PY{p}{)}
         \PY{n+nb}{print}\PY{p}{(}\PY{l+s+s2}{\PYZdq{}}\PY{l+s+s2}{b: }\PY{l+s+s2}{\PYZdq{}}\PY{p}{,} \PY{n}{b}\PY{p}{)}
\end{Verbatim}


    \begin{Verbatim}[commandchars=\\\{\}]
a:  [10 11 12 13 14 15 16 17 18]
b:  [1.         1.66666667 2.33333333 3.         3.66666667 4.33333333
 5.         5.66666667 6.33333333 7.        ]

    \end{Verbatim}

    For \texttt{a} and \texttt{b} above do the follow and print out the
results.

\begin{enumerate}
\def\labelenumi{\arabic{enumi}.}
\tightlist
\item
  Square all the elements in both arrays (element-wise).\\
\item
  Add both the squared arrays (e.g. {[}1,2{]} + {[}3,4{]} =
  {[}4,6{]}).\\
\item
  Sum the elements with even indices of the added array.\\
\item
  Take the square root of the added array (element-wise square root).
\end{enumerate}

    \begin{Verbatim}[commandchars=\\\{\}]
{\color{incolor}In [{\color{incolor}27}]:} \PY{n}{sq\PYZus{}a} \PY{o}{=} \PY{n}{np}\PY{o}{.}\PY{n}{square}\PY{p}{(}\PY{n}{a}\PY{p}{)}
         \PY{n}{sq\PYZus{}b} \PY{o}{=} \PY{n}{np}\PY{o}{.}\PY{n}{square}\PY{p}{(}\PY{n}{b}\PY{p}{)}
         \PY{n}{sum\PYZus{}ab} \PY{o}{=} \PY{n}{np}\PY{o}{.}\PY{n}{add}\PY{p}{(}\PY{n}{sq\PYZus{}a}\PY{p}{,} \PY{n}{sq\PYZus{}b}\PY{p}{)}
         \PY{n}{even\PYZus{}total} \PY{o}{=} \PY{l+m+mi}{0}
         \PY{k}{for} \PY{n}{i} \PY{o+ow}{in} \PY{n+nb}{range}\PY{p}{(}\PY{n+nb}{len}\PY{p}{(}\PY{n}{sum\PYZus{}ab}\PY{p}{)}\PY{p}{)}\PY{p}{:}
             \PY{k}{if} \PY{n}{i}\PY{o}{\PYZpc{}}\PY{k}{2} == 0:
                 \PY{n}{even\PYZus{}total} \PY{o}{+}\PY{o}{=} \PY{n}{sum\PYZus{}ab}\PY{p}{[}\PY{n}{i}\PY{p}{]}
         \PY{n}{sqrt\PYZus{}ab} \PY{o}{=} \PY{n}{np}\PY{o}{.}\PY{n}{sqrt}\PY{p}{(}\PY{n}{sum\PYZus{}ab}\PY{p}{)}
         
         \PY{n+nb}{print}\PY{p}{(}\PY{l+s+s2}{\PYZdq{}}\PY{l+s+s2}{1. }\PY{l+s+s2}{\PYZdq{}}\PY{p}{,} \PY{n}{sq\PYZus{}a}\PY{p}{,} \PY{n}{sq\PYZus{}b} \PY{p}{)}
         \PY{n+nb}{print}\PY{p}{(}\PY{l+s+s2}{\PYZdq{}}\PY{l+s+s2}{2. }\PY{l+s+s2}{\PYZdq{}}\PY{p}{,} \PY{n}{sum\PYZus{}ab}\PY{p}{)}
         \PY{n+nb}{print}\PY{p}{(}\PY{l+s+s2}{\PYZdq{}}\PY{l+s+s2}{3. }\PY{l+s+s2}{\PYZdq{}}\PY{p}{,} \PY{n}{even\PYZus{}total}\PY{p}{)}
         \PY{n+nb}{print}\PY{p}{(}\PY{l+s+s2}{\PYZdq{}}\PY{l+s+s2}{4. }\PY{l+s+s2}{\PYZdq{}}\PY{p}{,} \PY{n}{sqrt\PYZus{}ab}\PY{p}{)}
\end{Verbatim}


    \begin{Verbatim}[commandchars=\\\{\}]
1.  [100. 121. 144. 169. 196. 225. 256. 289. 324. 361.] [ 1.          2.77777778  5.44444444  9.         13.44444444 18.77777778
 25.         32.11111111 40.11111111 49.        ]
2.  [101.         123.77777778 149.44444444 178.         209.44444444
 243.77777778 281.         321.11111111 364.11111111 410.        ]
3.  1105.0
4.  [10.04987562 11.12554618 12.22474721 13.34166406 14.47219556 15.61338457
 16.76305461 17.91957341 19.08169571 20.24845673]

    \end{Verbatim}

    Append \texttt{b} to \texttt{a}. Reshape the appended array so that it
is a 5x4, 2D-array and store the results in a variable called
\texttt{m}. Print \texttt{m}.

    \begin{Verbatim}[commandchars=\\\{\}]
{\color{incolor}In [{\color{incolor}37}]:} \PY{n}{m} \PY{o}{=} \PY{n}{np}\PY{o}{.}\PY{n}{append}\PY{p}{(}\PY{n}{a}\PY{p}{,} \PY{n}{b}\PY{p}{)}\PY{o}{.}\PY{n}{reshape}\PY{p}{(}\PY{l+m+mi}{5}\PY{p}{,}\PY{l+m+mi}{4}\PY{p}{)}
         
         \PY{n+nb}{print}\PY{p}{(}\PY{l+s+s2}{\PYZdq{}}\PY{l+s+s2}{m: }\PY{l+s+s2}{\PYZdq{}}\PY{p}{,} \PY{n}{m}\PY{p}{)}
\end{Verbatim}


    \begin{Verbatim}[commandchars=\\\{\}]
m:  [[10.         11.         12.         13.        ]
 [14.         15.         16.         17.        ]
 [18.         19.          1.          1.66666667]
 [ 2.33333333  3.          3.66666667  4.33333333]
 [ 5.          5.66666667  6.33333333  7.        ]]

    \end{Verbatim}

    Extract the second and the third column of the matrix \texttt{m}. Store
the resulting 5x2 matrix in a new variable called \texttt{m2}. Print
\texttt{m2}.

    \begin{Verbatim}[commandchars=\\\{\}]
{\color{incolor}In [{\color{incolor}39}]:} \PY{n}{m2} \PY{o}{=} \PY{n}{m}\PY{p}{[}\PY{p}{:}\PY{p}{,} \PY{l+m+mi}{1}\PY{p}{:}\PY{l+m+mi}{3}\PY{p}{]}
         
         \PY{n+nb}{print}\PY{p}{(}\PY{l+s+s2}{\PYZdq{}}\PY{l+s+s2}{m2: }\PY{l+s+s2}{\PYZdq{}}\PY{p}{,} \PY{n}{m2}\PY{p}{)}
\end{Verbatim}


    \begin{Verbatim}[commandchars=\\\{\}]
m2:  [[11.         12.        ]
 [15.         16.        ]
 [19.          1.        ]
 [ 3.          3.66666667]
 [ 5.66666667  6.33333333]]

    \end{Verbatim}

    Take the dot product of \texttt{m2} and \texttt{m} store the results in
a matrix called \texttt{m3}. Print \texttt{m3}. Note that dot product of
two matrices \(A\cdot B = A^T B\)

    \begin{Verbatim}[commandchars=\\\{\}]
{\color{incolor}In [{\color{incolor}45}]:} \PY{n}{m3} \PY{o}{=} \PY{n}{np}\PY{o}{.}\PY{n}{matmul}\PY{p}{(}\PY{n}{m2}\PY{o}{.}\PY{n}{transpose}\PY{p}{(}\PY{p}{)}\PY{p}{,} \PY{n}{m}\PY{p}{)}
         
         \PY{n+nb}{print}\PY{p}{(}\PY{l+s+s2}{\PYZdq{}}\PY{l+s+s2}{m3: }\PY{l+s+s2}{\PYZdq{}}\PY{p}{,} \PY{n}{m3}\PY{p}{)}
\end{Verbatim}


    \begin{Verbatim}[commandchars=\\\{\}]
m3:  [[697.33333333 748.11111111 437.88888889 482.33333333]
 [402.22222222 437.88888889 454.55555556 489.88888889]]

    \end{Verbatim}

    Round the \texttt{m3} matrix to two decimal points. Store the result in
place and print the new \texttt{m3}.

    \begin{Verbatim}[commandchars=\\\{\}]
{\color{incolor}In [{\color{incolor}47}]:} \PY{n}{m3} \PY{o}{=} \PY{n}{np}\PY{o}{.}\PY{n}{round}\PY{p}{(}\PY{n}{m3}\PY{p}{,} \PY{l+m+mi}{2}\PY{p}{)}
         
         \PY{n+nb}{print}\PY{p}{(}\PY{l+s+s2}{\PYZdq{}}\PY{l+s+s2}{m3: }\PY{l+s+s2}{\PYZdq{}}\PY{p}{,} \PY{n}{m3}\PY{p}{)}
\end{Verbatim}


    \begin{Verbatim}[commandchars=\\\{\}]
m3:  [[697.33 748.11 437.89 482.33]
 [402.22 437.89 454.56 489.89]]

    \end{Verbatim}

    Sort the \texttt{m3} array so that the highest value is at the top left,
the next highest value to the right of the highest, and the lowest value
is at the bottom right. Print the sorted \texttt{m3} array.

    \begin{Verbatim}[commandchars=\\\{\}]
{\color{incolor}In [{\color{incolor}48}]:} \PY{n}{sorted\PYZus{}m3} \PY{o}{=} \PY{n}{np}\PY{o}{.}\PY{n}{sort}\PY{p}{(}\PY{n}{m3}\PY{p}{)}
         
         \PY{n+nb}{print}\PY{p}{(}\PY{l+s+s2}{\PYZdq{}}\PY{l+s+s2}{sorted m3: }\PY{l+s+s2}{\PYZdq{}}\PY{p}{,} \PY{n}{sorted\PYZus{}m3}\PY{p}{)}
\end{Verbatim}


    \begin{Verbatim}[commandchars=\\\{\}]
sorted m3:  [[437.89 482.33 697.33 748.11]
 [402.22 437.89 454.56 489.89]]

    \end{Verbatim}

    \hypertarget{numpy-and-masks}{%
\subsection{NumPy and Masks}\label{numpy-and-masks}}

    Create an array called \texttt{f} where there are 100 equally-spaced
values from 0 to pi, inclusive. Take the sin of the array \texttt{f}
(element-wise) and store that in place. Print \texttt{f}.

    \begin{Verbatim}[commandchars=\\\{\}]
{\color{incolor}In [{\color{incolor}62}]:} \PY{n}{f} \PY{o}{=} \PY{n}{np}\PY{o}{.}\PY{n}{linspace}\PY{p}{(}\PY{l+m+mi}{0}\PY{p}{,} \PY{n}{math}\PY{o}{.}\PY{n}{pi}\PY{p}{,} \PY{l+m+mi}{100}\PY{p}{)}
         \PY{n}{f} \PY{o}{=} \PY{n}{np}\PY{o}{.}\PY{n}{sin}\PY{p}{(}\PY{n}{f}\PY{p}{)}
         
         \PY{n+nb}{print}\PY{p}{(}\PY{l+s+s2}{\PYZdq{}}\PY{l+s+s2}{f: }\PY{l+s+s2}{\PYZdq{}}\PY{p}{,} \PY{n}{f}\PY{p}{)}
\end{Verbatim}


    \begin{Verbatim}[commandchars=\\\{\}]
f:  [0.00000000e+00 3.17279335e-02 6.34239197e-02 9.50560433e-02
 1.26592454e-01 1.58001396e-01 1.89251244e-01 2.20310533e-01
 2.51147987e-01 2.81732557e-01 3.12033446e-01 3.42020143e-01
 3.71662456e-01 4.00930535e-01 4.29794912e-01 4.58226522e-01
 4.86196736e-01 5.13677392e-01 5.40640817e-01 5.67059864e-01
 5.92907929e-01 6.18158986e-01 6.42787610e-01 6.66769001e-01
 6.90079011e-01 7.12694171e-01 7.34591709e-01 7.55749574e-01
 7.76146464e-01 7.95761841e-01 8.14575952e-01 8.32569855e-01
 8.49725430e-01 8.66025404e-01 8.81453363e-01 8.95993774e-01
 9.09631995e-01 9.22354294e-01 9.34147860e-01 9.45000819e-01
 9.54902241e-01 9.63842159e-01 9.71811568e-01 9.78802446e-01
 9.84807753e-01 9.89821442e-01 9.93838464e-01 9.96854776e-01
 9.98867339e-01 9.99874128e-01 9.99874128e-01 9.98867339e-01
 9.96854776e-01 9.93838464e-01 9.89821442e-01 9.84807753e-01
 9.78802446e-01 9.71811568e-01 9.63842159e-01 9.54902241e-01
 9.45000819e-01 9.34147860e-01 9.22354294e-01 9.09631995e-01
 8.95993774e-01 8.81453363e-01 8.66025404e-01 8.49725430e-01
 8.32569855e-01 8.14575952e-01 7.95761841e-01 7.76146464e-01
 7.55749574e-01 7.34591709e-01 7.12694171e-01 6.90079011e-01
 6.66769001e-01 6.42787610e-01 6.18158986e-01 5.92907929e-01
 5.67059864e-01 5.40640817e-01 5.13677392e-01 4.86196736e-01
 4.58226522e-01 4.29794912e-01 4.00930535e-01 3.71662456e-01
 3.42020143e-01 3.12033446e-01 2.81732557e-01 2.51147987e-01
 2.20310533e-01 1.89251244e-01 1.58001396e-01 1.26592454e-01
 9.50560433e-02 6.34239197e-02 3.17279335e-02 1.22464680e-16]

    \end{Verbatim}

    Use a `mask' and print an array that is True when \texttt{f}
\textgreater{}= 1/2 and False when \texttt{f} \textless{} 1/2. Print an
array sequence that has only those values where \texttt{f}
\textgreater{}= 1/2.

    \begin{Verbatim}[commandchars=\\\{\}]
{\color{incolor}In [{\color{incolor}80}]:} \PY{n}{m} \PY{o}{=} \PY{p}{[}\PY{n}{elem} \PY{o}{\PYZgt{}}\PY{o}{=} \PY{l+m+mf}{0.5} \PY{k}{for} \PY{n}{elem} \PY{o+ow}{in} \PY{n}{f}\PY{p}{]}
         \PY{n}{f}\PY{p}{[}\PY{n}{m}\PY{p}{]}
\end{Verbatim}


\begin{Verbatim}[commandchars=\\\{\}]
{\color{outcolor}Out[{\color{outcolor}80}]:} array([0.51367739, 0.54064082, 0.56705986, 0.59290793, 0.61815899,
                0.64278761, 0.666769  , 0.69007901, 0.71269417, 0.73459171,
                0.75574957, 0.77614646, 0.79576184, 0.81457595, 0.83256985,
                0.84972543, 0.8660254 , 0.88145336, 0.89599377, 0.909632  ,
                0.92235429, 0.93414786, 0.94500082, 0.95490224, 0.96384216,
                0.97181157, 0.97880245, 0.98480775, 0.98982144, 0.99383846,
                0.99685478, 0.99886734, 0.99987413, 0.99987413, 0.99886734,
                0.99685478, 0.99383846, 0.98982144, 0.98480775, 0.97880245,
                0.97181157, 0.96384216, 0.95490224, 0.94500082, 0.93414786,
                0.92235429, 0.909632  , 0.89599377, 0.88145336, 0.8660254 ,
                0.84972543, 0.83256985, 0.81457595, 0.79576184, 0.77614646,
                0.75574957, 0.73459171, 0.71269417, 0.69007901, 0.666769  ,
                0.64278761, 0.61815899, 0.59290793, 0.56705986, 0.54064082,
                0.51367739])
\end{Verbatim}
            
    \hypertarget{numpy-and-2-variable-prediction}{%
\subsection{NumPy and 2 Variable
Prediction}\label{numpy-and-2-variable-prediction}}

    Let \texttt{x} be the number of miles a person drives per day and
\texttt{y} be the dollars spent on buying car fuel per day.

We have created 2 numpy arrays each of size 100 that represent
\texttt{x} and \texttt{y}.\\
\texttt{x} (number of miles) ranges from 1 to 10 with a uniform noise of
(0, 1/2).\\
\texttt{y} (money spent in dollars) will be from 1 to 20 with a uniform
noise (0, 1).

Run the cell below.

    \begin{Verbatim}[commandchars=\\\{\}]
{\color{incolor}In [{\color{incolor}67}]:} \PY{c+c1}{\PYZsh{} seed the random number generator with a fixed value}
         \PY{n}{np}\PY{o}{.}\PY{n}{random}\PY{o}{.}\PY{n}{seed}\PY{p}{(}\PY{l+m+mi}{500}\PY{p}{)}
         
         \PY{n}{x}\PY{o}{=}\PY{n}{np}\PY{o}{.}\PY{n}{linspace}\PY{p}{(}\PY{l+m+mi}{1}\PY{p}{,}\PY{l+m+mi}{10}\PY{p}{,}\PY{l+m+mi}{100}\PY{p}{)}\PY{o}{+} \PY{n}{np}\PY{o}{.}\PY{n}{random}\PY{o}{.}\PY{n}{uniform}\PY{p}{(}\PY{n}{low}\PY{o}{=}\PY{l+m+mi}{0}\PY{p}{,}\PY{n}{high}\PY{o}{=}\PY{o}{.}\PY{l+m+mi}{5}\PY{p}{,}\PY{n}{size}\PY{o}{=}\PY{l+m+mi}{100}\PY{p}{)} 
         \PY{n}{y}\PY{o}{=}\PY{n}{np}\PY{o}{.}\PY{n}{linspace}\PY{p}{(}\PY{l+m+mi}{1}\PY{p}{,}\PY{l+m+mi}{20}\PY{p}{,}\PY{l+m+mi}{100}\PY{p}{)}\PY{o}{+} \PY{n}{np}\PY{o}{.}\PY{n}{random}\PY{o}{.}\PY{n}{uniform}\PY{p}{(}\PY{n}{low}\PY{o}{=}\PY{l+m+mi}{0}\PY{p}{,}\PY{n}{high}\PY{o}{=}\PY{l+m+mi}{1}\PY{p}{,}\PY{n}{size}\PY{o}{=}\PY{l+m+mi}{100}\PY{p}{)}
         \PY{n+nb}{print} \PY{p}{(}\PY{l+s+s1}{\PYZsq{}}\PY{l+s+s1}{x = }\PY{l+s+s1}{\PYZsq{}}\PY{p}{,}\PY{n}{x}\PY{p}{)}
         \PY{n+nb}{print} \PY{p}{(}\PY{l+s+s1}{\PYZsq{}}\PY{l+s+s1}{y= }\PY{l+s+s1}{\PYZsq{}}\PY{p}{,}\PY{n}{y}\PY{p}{)}
\end{Verbatim}


    \begin{Verbatim}[commandchars=\\\{\}]
x =  [ 1.34683976  1.12176759  1.51512398  1.55233174  1.40619168  1.65075498
  1.79399331  1.80243817  1.89844195  2.00100023  2.3344038   2.22424872
  2.24914511  2.36268477  2.49808849  2.8212704   2.68452475  2.68229427
  3.09511169  2.95703884  3.09047742  3.2544361   3.41541904  3.40886375
  3.50672677  3.74960644  3.64861355  3.7721462   3.56368566  4.01092701
  4.15630694  4.06088549  4.02517179  4.25169402  4.15897504  4.26835333
  4.32520644  4.48563164  4.78490721  4.84614839  4.96698768  5.18754259
  5.29582013  5.32097781  5.0674106   5.47601124  5.46852704  5.64537452
  5.49642807  5.89755027  5.68548923  5.76276141  5.94613234  6.18135713
  5.96522091  6.0275473   6.54290191  6.4991329   6.74003765  6.81809807
  6.50611821  6.91538752  7.01250925  6.89905417  7.31314433  7.20472297
  7.1043621   7.48199528  7.58957227  7.61744354  7.6991707   7.85436822
  8.03510784  7.80787781  8.22410224  7.99366248  8.40581097  8.28913792
  8.45971515  8.54227144  8.6906456   8.61856507  8.83489887  8.66309658
  8.94837987  9.20890222  8.9614749   8.92608294  9.13231416  9.55889896
  9.61488451  9.54252979  9.42015491  9.90952569 10.00659591 10.02504265
 10.07330937  9.93489915 10.0892334  10.36509991]
y=  [ 1.6635012   2.0214592   2.10816052  2.26016496  1.96287558  2.9554635
  3.02881887  3.33565296  2.75465779  3.4250107   3.39670148  3.39377767
  3.78503343  4.38293049  4.32963586  4.03925039  4.73691868  4.30098399
  4.8416329   4.78175957  4.99765787  5.31746817  5.76844671  5.93723749
  5.72811642  6.70973615  6.68143367  6.57482731  7.17737603  7.54863252
  7.30221419  7.3202573   7.78023884  7.91133365  8.2765417   8.69203281
  8.78219865  8.45897546  8.89094715  8.81719921  8.87106971  9.66192562
  9.4020625   9.85990783  9.60359778 10.07386266 10.6957995  10.66721916
 11.18256285 10.57431836 11.46744716 10.94398916 11.26445259 12.09754828
 12.11988037 12.121557   12.17613693 12.43750193 13.00912372 12.86407194
 13.24640866 12.76120085 13.11723062 14.07841099 14.19821707 14.27289001
 14.30624942 14.63060835 14.2770918  15.0744923  14.45261619 15.11897313
 15.2378667  15.27203124 15.32491892 16.01095271 15.71250558 16.29488506
 16.70618934 16.56555394 16.42379457 17.18144744 17.13813976 17.69613625
 17.37763019 17.90942839 17.90343733 18.01951169 18.35727914 18.16841269
 18.61813748 18.66062754 18.81217983 19.44995194 19.7213867  19.71966726
 19.78961904 19.64385088 20.69719809 20.07974319]

    \end{Verbatim}

    Find the expected value of \texttt{x} and the expected value of
\texttt{y}.

    \begin{Verbatim}[commandchars=\\\{\}]
{\color{incolor}In [{\color{incolor}68}]:} \PY{n}{ex\PYZus{}x} \PY{o}{=} \PY{n}{np}\PY{o}{.}\PY{n}{mean}\PY{p}{(}\PY{n}{x}\PY{p}{)}
         \PY{n}{ex\PYZus{}y} \PY{o}{=} \PY{n}{np}\PY{o}{.}\PY{n}{mean}\PY{p}{(}\PY{n}{y}\PY{p}{)}
         
         \PY{n+nb}{print} \PY{p}{(}\PY{l+s+s1}{\PYZsq{}}\PY{l+s+s1}{expected value of x = }\PY{l+s+s1}{\PYZsq{}}\PY{p}{,} \PY{n}{ex\PYZus{}x}\PY{p}{)}
         \PY{n+nb}{print} \PY{p}{(}\PY{l+s+s1}{\PYZsq{}}\PY{l+s+s1}{expected value of y = }\PY{l+s+s1}{\PYZsq{}}\PY{p}{,} \PY{n}{ex\PYZus{}y}\PY{p}{)}
\end{Verbatim}


    \begin{Verbatim}[commandchars=\\\{\}]
expected value of x =  5.782532541587923
expected value of y =  11.012981683344968

    \end{Verbatim}

    Find the variance for \texttt{x} and \texttt{y}.

    \begin{Verbatim}[commandchars=\\\{\}]
{\color{incolor}In [{\color{incolor} }]:} \PY{n}{var\PYZus{}x} \PY{o}{=} \PY{n}{np}\PY{o}{.}\PY{n}{var}\PY{p}{(}\PY{n}{x}\PY{p}{)}
        \PY{n}{var\PYZus{}y} \PY{o}{=} \PY{n}{np}\PY{o}{.}\PY{n}{var}\PY{p}{(}\PY{n}{y}\PY{p}{)}
        
        \PY{n+nb}{print} \PY{p}{(}\PY{l+s+s1}{\PYZsq{}}\PY{l+s+s1}{Variance of x = }\PY{l+s+s1}{\PYZsq{}}\PY{p}{,} \PY{n}{ex\PYZus{}x}\PY{p}{)}
        \PY{n+nb}{print} \PY{p}{(}\PY{l+s+s1}{\PYZsq{}}\PY{l+s+s1}{Variance of y = }\PY{l+s+s1}{\PYZsq{}}\PY{p}{,} \PY{n}{ex\PYZus{}y}\PY{p}{)}
\end{Verbatim}


    Find the co-variance of \texttt{x} and \texttt{y}.

    \begin{Verbatim}[commandchars=\\\{\}]
{\color{incolor}In [{\color{incolor}69}]:} \PY{n}{cov\PYZus{}xy} \PY{o}{=} \PY{n}{np}\PY{o}{.}\PY{n}{cov}\PY{p}{(}\PY{n}{x}\PY{p}{,} \PY{n}{y}\PY{p}{)}
         
         \PY{n+nb}{print}\PY{p}{(}\PY{l+s+s1}{\PYZsq{}}\PY{l+s+s1}{Covariance of x and y = }\PY{l+s+s1}{\PYZsq{}}\PY{p}{,} \PY{n}{cov\PYZus{}xy}\PY{p}{)}
\end{Verbatim}


    \begin{Verbatim}[commandchars=\\\{\}]
Covariance of x and y =  [[ 7.10437124 14.65774383]
 [14.65774383 30.41808442]]

    \end{Verbatim}

    Assume that the number of dollars spent on car fuel is only linearly
dependent on the miles driven. Write code that uses a linear predictor
to calculate a predicted value of \texttt{y} for each \texttt{x}.

i.e. \(y_{predicted} = f(x) = mx + b.\)

    \begin{Verbatim}[commandchars=\\\{\}]
{\color{incolor}In [{\color{incolor}71}]:} \PY{n}{A} \PY{o}{=} \PY{n}{np}\PY{o}{.}\PY{n}{vstack}\PY{p}{(}\PY{p}{[}\PY{n}{x}\PY{p}{,} \PY{n}{np}\PY{o}{.}\PY{n}{ones}\PY{p}{(}\PY{n+nb}{len}\PY{p}{(}\PY{n}{x}\PY{p}{)}\PY{p}{)}\PY{p}{]}\PY{p}{)}\PY{o}{.}\PY{n}{transpose}\PY{p}{(}\PY{p}{)}
         \PY{n}{result} \PY{o}{=} \PY{n}{np}\PY{o}{.}\PY{n}{linalg}\PY{o}{.}\PY{n}{lstsq}\PY{p}{(}\PY{n}{A}\PY{p}{,} \PY{n}{y}\PY{p}{,} \PY{n}{rcond}\PY{o}{=}\PY{k+kc}{None}\PY{p}{)}
         \PY{n}{m}\PY{p}{,} \PY{n}{b} \PY{o}{=} \PY{n}{result}\PY{p}{[}\PY{l+m+mi}{0}\PY{p}{]}
         
         \PY{n}{y\PYZus{}predicted} \PY{o}{=} \PY{n+nb}{str}\PY{p}{(}\PY{n}{m}\PY{p}{)} \PY{o}{+} \PY{l+s+s1}{\PYZsq{}}\PY{l+s+s1}{x + }\PY{l+s+s1}{\PYZsq{}} \PY{o}{+} \PY{n+nb}{str}\PY{p}{(}\PY{n}{b}\PY{p}{)}
         \PY{n+nb}{print}\PY{p}{(}\PY{n}{y\PYZus{}predicted}\PY{p}{)}
\end{Verbatim}


    \begin{Verbatim}[commandchars=\\\{\}]
2.063200715971325x + -0.9175435965867224

    \end{Verbatim}

    Predict \texttt{y} for each value in \texttt{x}, put the error into an
array called \(y_{error}\).

    \begin{Verbatim}[commandchars=\\\{\}]
{\color{incolor}In [{\color{incolor}72}]:} \PY{n}{predict} \PY{o}{=} \PY{n}{m}\PY{o}{*}\PY{n}{x} \PY{o}{+} \PY{n}{b}
         \PY{n}{y\PYZus{}error} \PY{o}{=} \PY{n}{y} \PY{o}{\PYZhy{}} \PY{n}{predict}
         
         \PY{n+nb}{print}\PY{p}{(}\PY{n}{y\PYZus{}error}\PY{p}{)}
\end{Verbatim}


    \begin{Verbatim}[commandchars=\\\{\}]
[-0.19775597  0.62457111 -0.10030076 -0.02506341 -0.02083649  0.46716823
  0.24499418  0.53440482 -0.24466541  0.21408918 -0.50209852 -0.27775029
  0.06213923  0.42578118  0.0931215  -0.86405311  0.1157489  -0.31558388
 -0.62666017 -0.40166149 -0.46107377 -0.47954311 -0.3607047  -0.17838904
 -0.58942116 -0.10891094  0.07115518 -0.29032384  0.74232081  0.19082863
 -0.35553767 -0.14062095  0.39304511  0.0567791   0.61328502  0.80310676
  0.77597321  0.12176065 -0.06373323 -0.26383402 -0.45927925 -0.12347238
 -0.60673379 -0.20079382  0.0660562  -0.30670405  0.33067419 -0.062778
  0.75987212 -0.67596798  0.65468531 -0.02820071 -0.08606832  0.26171143
  0.72997592  0.60306068 -0.40563939 -0.05397013  0.02061681 -0.28548928
  0.7405245  -0.58908804 -0.43343988  0.76182107  0.02727604  0.32564401
  0.56606805  0.11129392 -0.46417555  0.27572093 -0.5147747  -0.16862142
 -0.42262995  0.08035574 -0.72551112  0.43596616 -0.71282602  0.11027337
  0.16964259 -0.14132301 -0.58920807  0.31716141 -0.17248631  0.73997278
 -0.16712997 -0.17284167  0.33165948  0.52075457  0.43302563 -0.6359709
 -0.30175553 -0.10998314  0.29405306 -0.07784496 -0.00668554 -0.04646431
 -0.07609646  0.06370343  0.79862812 -0.38799477]

    \end{Verbatim}

    Write code that calculates the root mean square error (RMSE).

    \begin{Verbatim}[commandchars=\\\{\}]
{\color{incolor}In [{\color{incolor}73}]:} \PY{n}{rmse} \PY{o}{=} \PY{n}{np}\PY{o}{.}\PY{n}{sqrt}\PY{p}{(}\PY{n}{np}\PY{o}{.}\PY{n}{sum}\PY{p}{(}\PY{n}{y\PYZus{}error}\PY{o}{*}\PY{o}{*}\PY{l+m+mi}{2}\PY{p}{)}\PY{p}{)}
         \PY{n+nb}{print}\PY{p}{(}\PY{l+s+s1}{\PYZsq{}}\PY{l+s+s1}{RSME:}\PY{l+s+s1}{\PYZsq{}}\PY{p}{,} \PY{n}{rmse}\PY{p}{)}
\end{Verbatim}


    \begin{Verbatim}[commandchars=\\\{\}]
RSME: 4.176777236685611

    \end{Verbatim}

    \hypertarget{pandas}{%
\subsection{Pandas}\label{pandas}}

    \hypertarget{reading-a-file}{%
\subsubsection{Reading a File}\label{reading-a-file}}

    Read in a CSV file called `data3.csv' into a dataframe called df.

Data description * Data source:
http://www.fao.org/nr/water/aquastat/data/query/index.html * Data, units
* GDP, current USD (CPI adjusted) * NRI, mm/yr * Population density,
inhab/km\^{}2 * Total area of the country, 1000 ha = 10km\^{}2 * Total
Population, unit 1000 inhabitants

Display the first 10 lines of the dataframe.

    \begin{Verbatim}[commandchars=\\\{\}]
{\color{incolor}In [{\color{incolor}92}]:} \PY{n}{df} \PY{o}{=} \PY{n}{df} \PY{o}{=} \PY{n}{pd}\PY{o}{.}\PY{n}{read\PYZus{}csv}\PY{p}{(}\PY{l+s+s1}{\PYZsq{}}\PY{l+s+s1}{data3.csv}\PY{l+s+s1}{\PYZsq{}}\PY{p}{)}
         \PY{n}{df}\PY{o}{.}\PY{n}{head}\PY{p}{(}\PY{l+m+mi}{10}\PY{p}{)}
\end{Verbatim}


\begin{Verbatim}[commandchars=\\\{\}]
{\color{outcolor}Out[{\color{outcolor}92}]:}         Area  Area Id              Variable Name  Variable Id    Year  \textbackslash{}
         0  Argentina      9.0  Total area of the country       4100.0  1962.0   
         1  Argentina      9.0  Total area of the country       4100.0  1967.0   
         2  Argentina      9.0  Total area of the country       4100.0  1972.0   
         3  Argentina      9.0  Total area of the country       4100.0  1977.0   
         4  Argentina      9.0  Total area of the country       4100.0  1982.0   
         5  Argentina      9.0  Total area of the country       4100.0  1987.0   
         6  Argentina      9.0  Total area of the country       4100.0  1992.0   
         7  Argentina      9.0  Total area of the country       4100.0  1997.0   
         8  Argentina      9.0  Total area of the country       4100.0  2002.0   
         9  Argentina      9.0  Total area of the country       4100.0  2007.0   
         
               Value Symbol  Md  
         0  278040.0      E NaN  
         1  278040.0      E NaN  
         2  278040.0      E NaN  
         3  278040.0      E NaN  
         4  278040.0      E NaN  
         5  278040.0      E NaN  
         6  278040.0      E NaN  
         7  278040.0      E NaN  
         8  278040.0      E NaN  
         9  278040.0      E NaN  
\end{Verbatim}
            
    Display the column names.

    \begin{Verbatim}[commandchars=\\\{\}]
{\color{incolor}In [{\color{incolor}93}]:} \PY{n+nb}{print}\PY{p}{(}\PY{n}{df}\PY{o}{.}\PY{n}{columns}\PY{p}{)}
\end{Verbatim}


    \begin{Verbatim}[commandchars=\\\{\}]
Index(['Area', 'Area Id', 'Variable Name', 'Variable Id', 'Year', 'Value',
       'Symbol', 'Md'],
      dtype='object')

    \end{Verbatim}

    \hypertarget{data-preprocessing}{%
\subsubsection{Data Preprocessing}\label{data-preprocessing}}

    Create a mask of NAN values (i.e.~apply \texttt{.isnull} on the
dataframe). Inspect the mask for `True' values, they denote NANs.\\
\emph{Hint: You will notice that the last 8 rows and the last column
(`Other') have NAN values. You can also use df.tail() to see the last
row.}

Remove the bottom 8 rows from the dataframe because they contain NAN
values. Also remove the column `Other'.

    \begin{Verbatim}[commandchars=\\\{\}]
{\color{incolor}In [{\color{incolor}94}]:} \PY{n}{df}\PY{o}{.}\PY{n}{isnull}\PY{p}{(}\PY{p}{)}\PY{o}{.}\PY{n}{tail}\PY{p}{(}\PY{p}{)}
         \PY{n}{df}\PY{o}{.}\PY{n}{drop}\PY{p}{(}\PY{p}{[}\PY{l+s+s1}{\PYZsq{}}\PY{l+s+s1}{Md}\PY{l+s+s1}{\PYZsq{}}\PY{p}{]}\PY{p}{,} \PY{n}{axis}\PY{o}{=}\PY{l+m+mi}{1}\PY{p}{)}
         \PY{n}{df} \PY{o}{=} \PY{n}{df}\PY{p}{[}\PY{p}{:}\PY{o}{\PYZhy{}}\PY{l+m+mi}{8}\PY{p}{]}
\end{Verbatim}


    All the columns in our dataframe are not required for analysis. Drop
these columns: \texttt{Area\ Id}, \texttt{Variable\ Id}, and
\texttt{Symbol} and save the new dataframe as \texttt{df1}.

    \begin{Verbatim}[commandchars=\\\{\}]
{\color{incolor}In [{\color{incolor}99}]:} \PY{n}{df1} \PY{o}{=} \PY{n}{df}\PY{o}{.}\PY{n}{drop}\PY{p}{(}\PY{p}{[}\PY{l+s+s1}{\PYZsq{}}\PY{l+s+s1}{Area Id}\PY{l+s+s1}{\PYZsq{}}\PY{p}{,} \PY{l+s+s1}{\PYZsq{}}\PY{l+s+s1}{Variable Id}\PY{l+s+s1}{\PYZsq{}}\PY{p}{,} \PY{l+s+s1}{\PYZsq{}}\PY{l+s+s1}{Symbol}\PY{l+s+s1}{\PYZsq{}}\PY{p}{,} \PY{l+s+s1}{\PYZsq{}}\PY{l+s+s1}{Md}\PY{l+s+s1}{\PYZsq{}}\PY{p}{]}\PY{p}{,} \PY{n}{axis}\PY{o}{=}\PY{l+m+mi}{1}\PY{p}{)}
         \PY{n}{df1}
\end{Verbatim}


\begin{Verbatim}[commandchars=\\\{\}]
{\color{outcolor}Out[{\color{outcolor}99}]:}                          Area                  Variable Name    Year  \textbackslash{}
         0                   Argentina      Total area of the country  1962.0   
         1                   Argentina      Total area of the country  1967.0   
         2                   Argentina      Total area of the country  1972.0   
         3                   Argentina      Total area of the country  1977.0   
         4                   Argentina      Total area of the country  1982.0   
         5                   Argentina      Total area of the country  1987.0   
         6                   Argentina      Total area of the country  1992.0   
         7                   Argentina      Total area of the country  1997.0   
         8                   Argentina      Total area of the country  2002.0   
         9                   Argentina      Total area of the country  2007.0   
         10                  Argentina      Total area of the country  2012.0   
         11                  Argentina      Total area of the country  2014.0   
         12                  Argentina               Total population  1962.0   
         13                  Argentina               Total population  1967.0   
         14                  Argentina               Total population  1972.0   
         15                  Argentina               Total population  1977.0   
         16                  Argentina               Total population  1982.0   
         17                  Argentina               Total population  1987.0   
         18                  Argentina               Total population  1992.0   
         19                  Argentina               Total population  1997.0   
         20                  Argentina               Total population  2002.0   
         21                  Argentina               Total population  2007.0   
         22                  Argentina               Total population  2012.0   
         23                  Argentina               Total population  2015.0   
         24                  Argentina             Population density  1962.0   
         25                  Argentina             Population density  1967.0   
         26                  Argentina             Population density  1972.0   
         27                  Argentina             Population density  1977.0   
         28                  Argentina             Population density  1982.0   
         29                  Argentina             Population density  1987.0   
         ..                        {\ldots}                            {\ldots}     {\ldots}   
         360  United States of America             Population density  1972.0   
         361  United States of America             Population density  1977.0   
         362  United States of America             Population density  1982.0   
         363  United States of America             Population density  1987.0   
         364  United States of America             Population density  1992.0   
         365  United States of America             Population density  1997.0   
         366  United States of America             Population density  2002.0   
         367  United States of America             Population density  2007.0   
         368  United States of America             Population density  2012.0   
         369  United States of America             Population density  2015.0   
         370  United States of America   Gross Domestic Product (GDP)  1962.0   
         371  United States of America   Gross Domestic Product (GDP)  1967.0   
         372  United States of America   Gross Domestic Product (GDP)  1972.0   
         373  United States of America   Gross Domestic Product (GDP)  1977.0   
         374  United States of America   Gross Domestic Product (GDP)  1982.0   
         375  United States of America   Gross Domestic Product (GDP)  1987.0   
         376  United States of America   Gross Domestic Product (GDP)  1992.0   
         377  United States of America   Gross Domestic Product (GDP)  1997.0   
         378  United States of America   Gross Domestic Product (GDP)  2002.0   
         379  United States of America   Gross Domestic Product (GDP)  2007.0   
         380  United States of America   Gross Domestic Product (GDP)  2012.0   
         381  United States of America   Gross Domestic Product (GDP)  2015.0   
         382  United States of America  National Rainfall Index (NRI)  1965.0   
         383  United States of America  National Rainfall Index (NRI)  1969.0   
         384  United States of America  National Rainfall Index (NRI)  1974.0   
         385  United States of America  National Rainfall Index (NRI)  1981.0   
         386  United States of America  National Rainfall Index (NRI)  1984.0   
         387  United States of America  National Rainfall Index (NRI)  1992.0   
         388  United States of America  National Rainfall Index (NRI)  1996.0   
         389  United States of America  National Rainfall Index (NRI)  2002.0   
         
                     Value  
         0    2.780400e+05  
         1    2.780400e+05  
         2    2.780400e+05  
         3    2.780400e+05  
         4    2.780400e+05  
         5    2.780400e+05  
         6    2.780400e+05  
         7    2.780400e+05  
         8    2.780400e+05  
         9    2.780400e+05  
         10   2.780400e+05  
         11   2.780400e+05  
         12   2.128800e+04  
         13   2.293200e+04  
         14   2.478300e+04  
         15   2.687900e+04  
         16   2.899400e+04  
         17   3.132600e+04  
         18   3.365500e+04  
         19   3.583400e+04  
         20   3.788900e+04  
         21   3.997000e+04  
         22   4.209500e+04  
         23   4.341700e+04  
         24   7.656000e+00  
         25   8.248000e+00  
         26   8.913000e+00  
         27   9.667000e+00  
         28   1.043000e+01  
         29   1.127000e+01  
         ..            {\ldots}  
         360  2.214000e+01  
         361  2.317000e+01  
         362  2.430000e+01  
         363  2.549000e+01  
         364  2.678000e+01  
         365  2.834000e+01  
         366  2.995000e+01  
         367  3.132000e+01  
         368  3.202000e+01  
         369  3.273000e+01  
         370  6.050000e+11  
         371  8.620000e+11  
         372  1.280000e+12  
         373  2.090000e+12  
         374  3.340000e+12  
         375  4.870000e+12  
         376  6.540000e+12  
         377  8.610000e+12  
         378  1.100000e+13  
         379  1.450000e+13  
         380  1.620000e+13  
         381  1.790000e+13  
         382  9.285000e+02  
         383  9.522000e+02  
         384  1.008000e+03  
         385  9.492000e+02  
         386  9.746000e+02  
         387  1.020000e+03  
         388  1.005000e+03  
         389  9.387000e+02  
         
         [390 rows x 4 columns]
\end{Verbatim}
            
    Display all the unique values in your new dataframe for these columns:
\texttt{Area}, \texttt{Variable\ Name}, and \texttt{Year}.

Note the Countries and the Metrics (ie.recorded variables) represented
in your dataset. \emph{Hint: Use .unique( ) method.}

    \begin{Verbatim}[commandchars=\\\{\}]
{\color{incolor}In [{\color{incolor}102}]:} \PY{k}{for} \PY{n}{name} \PY{o+ow}{in} \PY{p}{[}\PY{l+s+s1}{\PYZsq{}}\PY{l+s+s1}{Area}\PY{l+s+s1}{\PYZsq{}}\PY{p}{,} \PY{l+s+s1}{\PYZsq{}}\PY{l+s+s1}{Variable Name}\PY{l+s+s1}{\PYZsq{}}\PY{p}{,} \PY{l+s+s1}{\PYZsq{}}\PY{l+s+s1}{Year}\PY{l+s+s1}{\PYZsq{}}\PY{p}{]}\PY{p}{:}
              \PY{n+nb}{print}\PY{p}{(}\PY{n}{df1}\PY{p}{[}\PY{n}{name}\PY{p}{]}\PY{o}{.}\PY{n}{unique}\PY{p}{(}\PY{p}{)}\PY{p}{,} \PY{l+s+s2}{\PYZdq{}}\PY{l+s+se}{\PYZbs{}n}\PY{l+s+s2}{\PYZdq{}}\PY{p}{)}
\end{Verbatim}


    \begin{Verbatim}[commandchars=\\\{\}]
['Argentina' 'Australia' 'Germany' 'Iceland' 'Ireland' 'Sweden'
 'United States of America'] 

['Total area of the country' 'Total population' 'Population density'
 'Gross Domestic Product (GDP)' 'National Rainfall Index (NRI)'] 

[1962. 1967. 1972. 1977. 1982. 1987. 1992. 1997. 2002. 2007. 2012. 2014.
 2015. 1963. 1970. 1974. 1978. 1984. 1990. 1964. 1981. 1985. 1996. 2001.
 1969. 1973. 1979. 1993. 1971. 1975. 1986. 1991. 1998. 2000. 1965. 1983.
 1988. 1995.] 


    \end{Verbatim}

    Convert the \texttt{Year} column string values to pandas datetime
objects, where only the year is specified.\\
\emph{Hint: df1{[}`Year'{]} =
pd.to\_datetime(pd.Series(df1{[}`Year'{]}).astype(int),format=`\%Y').dt.year}

Run df1.tail() to see part of the result.

    \begin{Verbatim}[commandchars=\\\{\}]
{\color{incolor}In [{\color{incolor}104}]:} \PY{n}{df1}\PY{p}{[}\PY{l+s+s1}{\PYZsq{}}\PY{l+s+s1}{Year}\PY{l+s+s1}{\PYZsq{}}\PY{p}{]} \PY{o}{=} \PY{n}{pd}\PY{o}{.}\PY{n}{to\PYZus{}datetime}\PY{p}{(}\PY{n}{pd}\PY{o}{.}\PY{n}{Series}\PY{p}{(}\PY{n}{df1}\PY{p}{[}\PY{l+s+s1}{\PYZsq{}}\PY{l+s+s1}{Year}\PY{l+s+s1}{\PYZsq{}}\PY{p}{]}\PY{p}{)}\PY{o}{.}\PY{n}{astype}\PY{p}{(}\PY{n+nb}{int}\PY{p}{)}\PY{p}{,}\PY{n+nb}{format}\PY{o}{=}\PY{l+s+s1}{\PYZsq{}}\PY{l+s+s1}{\PYZpc{}}\PY{l+s+s1}{Y}\PY{l+s+s1}{\PYZsq{}}\PY{p}{)}\PY{o}{.}\PY{n}{dt}\PY{o}{.}\PY{n}{year}
          \PY{n}{df1}\PY{o}{.}\PY{n}{tail}\PY{p}{(}\PY{p}{)}
\end{Verbatim}


\begin{Verbatim}[commandchars=\\\{\}]
{\color{outcolor}Out[{\color{outcolor}104}]:}                          Area                  Variable Name  Year   Value
          385  United States of America  National Rainfall Index (NRI)  1981   949.2
          386  United States of America  National Rainfall Index (NRI)  1984   974.6
          387  United States of America  National Rainfall Index (NRI)  1992  1020.0
          388  United States of America  National Rainfall Index (NRI)  1996  1005.0
          389  United States of America  National Rainfall Index (NRI)  2002   938.7
\end{Verbatim}
            
    \hypertarget{extracting-statistics}{%
\subsubsection{Extracting Statistics}\label{extracting-statistics}}

    Create a dataframe `dftemp' to store rows where the \texttt{Area} is
\texttt{Iceland}.

    \begin{Verbatim}[commandchars=\\\{\}]
{\color{incolor}In [{\color{incolor}106}]:} \PY{n}{dftemp} \PY{o}{=} \PY{n}{df1}\PY{p}{[}\PY{n}{df1}\PY{p}{[}\PY{l+s+s1}{\PYZsq{}}\PY{l+s+s1}{Area}\PY{l+s+s1}{\PYZsq{}}\PY{p}{]} \PY{o}{==} \PY{l+s+s1}{\PYZsq{}}\PY{l+s+s1}{Iceland}\PY{l+s+s1}{\PYZsq{}}\PY{p}{]}
          \PY{n}{dftemp}\PY{o}{.}\PY{n}{head}\PY{p}{(}\PY{p}{)}
\end{Verbatim}


\begin{Verbatim}[commandchars=\\\{\}]
{\color{outcolor}Out[{\color{outcolor}106}]:}         Area              Variable Name  Year    Value
          166  Iceland  Total area of the country  1962  10300.0
          167  Iceland  Total area of the country  1967  10300.0
          168  Iceland  Total area of the country  1972  10300.0
          169  Iceland  Total area of the country  1977  10300.0
          170  Iceland  Total area of the country  1982  10300.0
\end{Verbatim}
            
    Print the years when the National Rainfall Index (NRI) was
\texttt{\textgreater{}\ 950} or \texttt{\textless{}\ 900} in Iceland
using the dataframe you created in the previous question.

    \begin{Verbatim}[commandchars=\\\{\}]
{\color{incolor}In [{\color{incolor}142}]:} \PY{n}{dftemp}\PY{p}{[}\PY{p}{(}\PY{n}{dftemp}\PY{p}{[}\PY{l+s+s1}{\PYZsq{}}\PY{l+s+s1}{Variable Name}\PY{l+s+s1}{\PYZsq{}}\PY{p}{]} \PY{o}{==} \PY{l+s+s1}{\PYZsq{}}\PY{l+s+s1}{National Rainfall Index (NRI)}\PY{l+s+s1}{\PYZsq{}}\PY{p}{)} 
                 \PY{o}{\PYZam{}} \PY{p}{(}\PY{p}{(}\PY{n}{dftemp}\PY{p}{[}\PY{l+s+s1}{\PYZsq{}}\PY{l+s+s1}{Value}\PY{l+s+s1}{\PYZsq{}}\PY{p}{]} \PY{o}{\PYZgt{}} \PY{l+m+mi}{950}\PY{p}{)} \PY{o}{|} \PY{p}{(}\PY{n}{dftemp}\PY{p}{[}\PY{l+s+s1}{\PYZsq{}}\PY{l+s+s1}{Value}\PY{l+s+s1}{\PYZsq{}}\PY{p}{]} \PY{o}{\PYZlt{}} \PY{l+m+mi}{900}\PY{p}{)}\PY{p}{)}\PY{p}{]}
\end{Verbatim}


\begin{Verbatim}[commandchars=\\\{\}]
{\color{outcolor}Out[{\color{outcolor}142}]:}         Area                  Variable Name  Year   Value
          214  Iceland  National Rainfall Index (NRI)  1967   816.0
          215  Iceland  National Rainfall Index (NRI)  1971   963.2
          216  Iceland  National Rainfall Index (NRI)  1975  1010.0
          218  Iceland  National Rainfall Index (NRI)  1986   968.5
          219  Iceland  National Rainfall Index (NRI)  1991  1095.0
          220  Iceland  National Rainfall Index (NRI)  1997   993.2
\end{Verbatim}
            
    Get all the rows of df1 (from the preprocessed data section of this
notebook) where the \texttt{Area} is
\texttt{United\ States\ of\ America} and store that into a new dataframe
called \texttt{df\_usa}. Set the indices of the this dataframe to be the
\texttt{Year} column.\\
\emph{Hint: Use .set\_index()}

    \begin{Verbatim}[commandchars=\\\{\}]
{\color{incolor}In [{\color{incolor}149}]:} \PY{n}{df\PYZus{}usa} \PY{o}{=} \PY{n}{df1}\PY{p}{[}\PY{n}{df1}\PY{p}{[}\PY{l+s+s1}{\PYZsq{}}\PY{l+s+s1}{Area}\PY{l+s+s1}{\PYZsq{}}\PY{p}{]} \PY{o}{==} \PY{l+s+s1}{\PYZsq{}}\PY{l+s+s1}{United States of America}\PY{l+s+s1}{\PYZsq{}}\PY{p}{]}\PY{o}{.}\PY{n}{set\PYZus{}index}\PY{p}{(}\PY{l+s+s1}{\PYZsq{}}\PY{l+s+s1}{Year}\PY{l+s+s1}{\PYZsq{}}\PY{p}{)}
          \PY{n}{df\PYZus{}usa}\PY{o}{.}\PY{n}{head}\PY{p}{(}\PY{p}{)}
\end{Verbatim}


\begin{Verbatim}[commandchars=\\\{\}]
{\color{outcolor}Out[{\color{outcolor}149}]:}                           Area              Variable Name     Value
          Year                                                               
          1962  United States of America  Total area of the country  962909.0
          1967  United States of America  Total area of the country  962909.0
          1972  United States of America  Total area of the country  962909.0
          1977  United States of America  Total area of the country  962909.0
          1982  United States of America  Total area of the country  962909.0
\end{Verbatim}
            
    Pivot the dataframe so that the unique \texttt{Variable\ Name} entries
become the column entries. The dataframe values should be the ones in
the \texttt{Value} column. Do this by running the lines of code below.

    \begin{Verbatim}[commandchars=\\\{\}]
{\color{incolor}In [{\color{incolor}150}]:} \PY{n}{df\PYZus{}usa} \PY{o}{=} \PY{n}{df\PYZus{}usa}\PY{o}{.}\PY{n}{pivot}\PY{p}{(}\PY{n}{columns}\PY{o}{=}\PY{l+s+s1}{\PYZsq{}}\PY{l+s+s1}{Variable Name}\PY{l+s+s1}{\PYZsq{}}\PY{p}{,}\PY{n}{values}\PY{o}{=}\PY{l+s+s1}{\PYZsq{}}\PY{l+s+s1}{Value}\PY{l+s+s1}{\PYZsq{}}\PY{p}{)}
          \PY{n}{df\PYZus{}usa}\PY{o}{.}\PY{n}{head}\PY{p}{(}\PY{p}{)}
\end{Verbatim}


\begin{Verbatim}[commandchars=\\\{\}]
{\color{outcolor}Out[{\color{outcolor}150}]:} Variable Name  Gross Domestic Product (GDP)  National Rainfall Index (NRI)  \textbackslash{}
          Year                                                                         
          1962                           6.050000e+11                            NaN   
          1965                                    NaN                          928.5   
          1967                           8.620000e+11                            NaN   
          1969                                    NaN                          952.2   
          1972                           1.280000e+12                            NaN   
          
          Variable Name  Population density  Total area of the country  Total population  
          Year                                                                            
          1962                        19.93                   962909.0          191861.0  
          1965                          NaN                        NaN               NaN  
          1967                        21.16                   962909.0          203713.0  
          1969                          NaN                        NaN               NaN  
          1972                        22.14                   962909.0          213220.0  
\end{Verbatim}
            
    Rename the corresponding columns to
{[}`GDP',`NRI',`PD',`Area',`Population'{]}.

    \begin{Verbatim}[commandchars=\\\{\}]
{\color{incolor}In [{\color{incolor}151}]:} \PY{n}{df\PYZus{}usa}\PY{o}{.}\PY{n}{columns} \PY{o}{=} \PY{p}{[}\PY{l+s+s1}{\PYZsq{}}\PY{l+s+s1}{GDP}\PY{l+s+s1}{\PYZsq{}}\PY{p}{,}\PY{l+s+s1}{\PYZsq{}}\PY{l+s+s1}{NRI}\PY{l+s+s1}{\PYZsq{}}\PY{p}{,}\PY{l+s+s1}{\PYZsq{}}\PY{l+s+s1}{PD}\PY{l+s+s1}{\PYZsq{}}\PY{p}{,}\PY{l+s+s1}{\PYZsq{}}\PY{l+s+s1}{Area}\PY{l+s+s1}{\PYZsq{}}\PY{p}{,}\PY{l+s+s1}{\PYZsq{}}\PY{l+s+s1}{Population}\PY{l+s+s1}{\PYZsq{}}\PY{p}{]}
          \PY{n}{df\PYZus{}usa}\PY{o}{.}\PY{n}{head}\PY{p}{(}\PY{p}{)}
\end{Verbatim}


\begin{Verbatim}[commandchars=\\\{\}]
{\color{outcolor}Out[{\color{outcolor}151}]:}                GDP    NRI     PD      Area  Population
          Year                                                  
          1962  6.050000e+11    NaN  19.93  962909.0    191861.0
          1965           NaN  928.5    NaN       NaN         NaN
          1967  8.620000e+11    NaN  21.16  962909.0    203713.0
          1969           NaN  952.2    NaN       NaN         NaN
          1972  1.280000e+12    NaN  22.14  962909.0    213220.0
\end{Verbatim}
            
    Print the output of \texttt{df\_usa.isnull().sum()}. This gives us the
number of NAN values in each column. Replace the NAN values by 0, using
\texttt{df\_usa=df\_usa.fillna(0)}. Print the output of
\texttt{df\_usa.isnull().sum()} again.

    \begin{Verbatim}[commandchars=\\\{\}]
{\color{incolor}In [{\color{incolor}152}]:} \PY{n+nb}{print}\PY{p}{(}\PY{l+s+s2}{\PYZdq{}}\PY{l+s+s2}{Number of NAN values before:}\PY{l+s+se}{\PYZbs{}n}\PY{l+s+s2}{\PYZdq{}}\PY{p}{,} \PY{n}{df\PYZus{}usa}\PY{o}{.}\PY{n}{isnull}\PY{p}{(}\PY{p}{)}\PY{o}{.}\PY{n}{sum}\PY{p}{(}\PY{p}{)}\PY{p}{)}
          
          \PY{n}{df\PYZus{}usa}\PY{o}{=}\PY{n}{df\PYZus{}usa}\PY{o}{.}\PY{n}{fillna}\PY{p}{(}\PY{l+m+mi}{0}\PY{p}{)}
          \PY{n+nb}{print}\PY{p}{(}\PY{l+s+s2}{\PYZdq{}}\PY{l+s+s2}{Number of NAN values after:}\PY{l+s+se}{\PYZbs{}n}\PY{l+s+s2}{ }\PY{l+s+s2}{\PYZdq{}}\PY{p}{,} \PY{n}{df\PYZus{}usa}\PY{o}{.}\PY{n}{isnull}\PY{p}{(}\PY{p}{)}\PY{o}{.}\PY{n}{sum}\PY{p}{(}\PY{p}{)}\PY{p}{)}
\end{Verbatim}


    \begin{Verbatim}[commandchars=\\\{\}]
Number of NAN values before:
 GDP            7
NRI           11
PD             7
Area           7
Population     7
dtype: int64
Number of NAN values after:
  GDP           0
NRI           0
PD            0
Area          0
Population    0
dtype: int64

    \end{Verbatim}

    Calculate and print all the column averages and the column standard
deviations.

    \begin{Verbatim}[commandchars=\\\{\}]
{\color{incolor}In [{\color{incolor}153}]:} \PY{n}{df\PYZus{}usa}\PY{o}{.}\PY{n}{describe}\PY{p}{(}\PY{p}{)}
\end{Verbatim}


\begin{Verbatim}[commandchars=\\\{\}]
{\color{outcolor}Out[{\color{outcolor}153}]:}                 GDP          NRI         PD           Area     Population
          count  1.900000e+01    19.000000  19.000000      19.000000      19.000000
          mean   4.620895e+12   409.273684  16.701579  610314.736842  161513.421053
          std    6.088656e+12   493.551503  13.554620  478948.168858  131380.538153
          min    0.000000e+00     0.000000   0.000000       0.000000       0.000000
          25\%    0.000000e+00     0.000000   0.000000       0.000000       0.000000
          50\%    1.280000e+12     0.000000  22.140000  962909.000000  213220.000000
          75\%    7.575000e+12   950.700000  27.560000  962909.000000  265395.500000
          max    1.790000e+13  1020.000000  32.730000  983151.000000  321774.000000
\end{Verbatim}
            
    Using the \texttt{df\_usa} dataframe, multiply the \texttt{Area} by 10
(so instead of 1000 ha, the unit becomes 100 ha = 1km\^{}2). Store the
result in place.

    \begin{Verbatim}[commandchars=\\\{\}]
{\color{incolor}In [{\color{incolor}154}]:} \PY{n}{df\PYZus{}usa}\PY{p}{[}\PY{l+s+s1}{\PYZsq{}}\PY{l+s+s1}{Area}\PY{l+s+s1}{\PYZsq{}}\PY{p}{]} \PY{o}{=} \PY{n}{df\PYZus{}usa}\PY{p}{[}\PY{l+s+s1}{\PYZsq{}}\PY{l+s+s1}{Area}\PY{l+s+s1}{\PYZsq{}}\PY{p}{]}\PY{o}{*}\PY{l+m+mi}{10}
          \PY{n}{df\PYZus{}usa}\PY{o}{.}\PY{n}{head}\PY{p}{(}\PY{p}{)}
\end{Verbatim}


\begin{Verbatim}[commandchars=\\\{\}]
{\color{outcolor}Out[{\color{outcolor}154}]:}                GDP    NRI     PD       Area  Population
          Year                                                   
          1962  6.050000e+11    0.0  19.93  9629090.0    191861.0
          1965  0.000000e+00  928.5   0.00        0.0         0.0
          1967  8.620000e+11    0.0  21.16  9629090.0    203713.0
          1969  0.000000e+00  952.2   0.00        0.0         0.0
          1972  1.280000e+12    0.0  22.14  9629090.0    213220.0
\end{Verbatim}
            
    Create a new column in \texttt{df\_usa} called \texttt{GDP/capita} and
populate it with the calculated GDP per capita. Round the results to two
decimal points. Store the result in place.

    \begin{Verbatim}[commandchars=\\\{\}]
{\color{incolor}In [{\color{incolor}159}]:} \PY{n}{df\PYZus{}usa}\PY{p}{[}\PY{l+s+s1}{\PYZsq{}}\PY{l+s+s1}{GDP/capita}\PY{l+s+s1}{\PYZsq{}}\PY{p}{]} \PY{o}{=} \PY{p}{(}\PY{n}{df\PYZus{}usa}\PY{p}{[}\PY{l+s+s1}{\PYZsq{}}\PY{l+s+s1}{GDP}\PY{l+s+s1}{\PYZsq{}}\PY{p}{]}\PY{o}{/}\PY{n}{df\PYZus{}usa}\PY{p}{[}\PY{l+s+s1}{\PYZsq{}}\PY{l+s+s1}{Population}\PY{l+s+s1}{\PYZsq{}}\PY{p}{]}\PY{p}{)}\PY{o}{.}\PY{n}{round}\PY{p}{(}\PY{l+m+mi}{2}\PY{p}{)}
          \PY{n}{df\PYZus{}usa}\PY{o}{.}\PY{n}{head}\PY{p}{(}\PY{p}{)}
\end{Verbatim}


\begin{Verbatim}[commandchars=\\\{\}]
{\color{outcolor}Out[{\color{outcolor}159}]:}                GDP    NRI     PD       Area  Population  GDP/capita       PD2
          Year                                                                         
          1962  6.050000e+11    0.0  19.93  9629090.0    191861.0  3153324.54  0.019925
          1965  0.000000e+00  928.5   0.00        0.0         0.0         NaN       NaN
          1967  8.620000e+11    0.0  21.16  9629090.0    203713.0  4231443.26  0.021156
          1969  0.000000e+00  952.2   0.00        0.0         0.0         NaN       NaN
          1972  1.280000e+12    0.0  22.14  9629090.0    213220.0  6003189.19  0.022143
\end{Verbatim}
            
    Create a new column in \texttt{df\_usa} called \texttt{PD2}
(i.e.~population density 2). Calculate the population density.
\textbf{Note: the units should be inhab/km\^{}2}. Round the reults to
two decimal point. Store the result in place.

    \begin{Verbatim}[commandchars=\\\{\}]
{\color{incolor}In [{\color{incolor}160}]:} \PY{n}{df\PYZus{}usa}\PY{p}{[}\PY{l+s+s1}{\PYZsq{}}\PY{l+s+s1}{PD2}\PY{l+s+s1}{\PYZsq{}}\PY{p}{]} \PY{o}{=} \PY{n}{df\PYZus{}usa}\PY{p}{[}\PY{l+s+s1}{\PYZsq{}}\PY{l+s+s1}{Population}\PY{l+s+s1}{\PYZsq{}}\PY{p}{]}\PY{o}{/}\PY{n}{df\PYZus{}usa}\PY{p}{[}\PY{l+s+s1}{\PYZsq{}}\PY{l+s+s1}{Area}\PY{l+s+s1}{\PYZsq{}}\PY{p}{]}\PY{o}{.}\PY{n}{round}\PY{p}{(}\PY{l+m+mi}{2}\PY{p}{)}
          \PY{n}{df\PYZus{}usa}\PY{o}{.}\PY{n}{head}\PY{p}{(}\PY{p}{)}
\end{Verbatim}


\begin{Verbatim}[commandchars=\\\{\}]
{\color{outcolor}Out[{\color{outcolor}160}]:}                GDP    NRI     PD       Area  Population  GDP/capita       PD2
          Year                                                                         
          1962  6.050000e+11    0.0  19.93  9629090.0    191861.0  3153324.54  0.019925
          1965  0.000000e+00  928.5   0.00        0.0         0.0         NaN       NaN
          1967  8.620000e+11    0.0  21.16  9629090.0    203713.0  4231443.26  0.021156
          1969  0.000000e+00  952.2   0.00        0.0         0.0         NaN       NaN
          1972  1.280000e+12    0.0  22.14  9629090.0    213220.0  6003189.19  0.022143
\end{Verbatim}
            
    Find the maximum value and minimum value of the `NRI' column in the USA
(using pandas methods). What years do the min and max values occur in?

    \begin{Verbatim}[commandchars=\\\{\}]
{\color{incolor}In [{\color{incolor}166}]:} \PY{n+nb}{print}\PY{p}{(}\PY{l+s+s1}{\PYZsq{}}\PY{l+s+s1}{Year of min NRI: =}\PY{l+s+s1}{\PYZsq{}}\PY{p}{,} \PY{n}{df\PYZus{}usa}\PY{p}{[}\PY{l+s+s1}{\PYZsq{}}\PY{l+s+s1}{NRI}\PY{l+s+s1}{\PYZsq{}}\PY{p}{]}\PY{o}{.}\PY{n}{idxmin}\PY{p}{(}\PY{p}{)}\PY{p}{)}
          \PY{n+nb}{print}\PY{p}{(}\PY{l+s+s1}{\PYZsq{}}\PY{l+s+s1}{Year of max MRI: =}\PY{l+s+s1}{\PYZsq{}}\PY{p}{,} \PY{n}{df\PYZus{}usa}\PY{p}{[}\PY{l+s+s1}{\PYZsq{}}\PY{l+s+s1}{NRI}\PY{l+s+s1}{\PYZsq{}}\PY{p}{]}\PY{o}{.}\PY{n}{idxmax}\PY{p}{(}\PY{p}{)}\PY{p}{)}
\end{Verbatim}


    \begin{Verbatim}[commandchars=\\\{\}]
Year of min NRI: = 1962
Year of max MRI: = 1992

    \end{Verbatim}

    \hypertarget{matplotlib}{%
\subsection{Matplotlib}\label{matplotlib}}

    Create a dataframe called \texttt{icecream} that has column
\texttt{Flavor} with entries \texttt{Strawberry}, \texttt{Vanilla}, and
\texttt{Chocolate} and another column with \texttt{Price} with entries
\texttt{3.50}, \texttt{3.00}, and \texttt{4.25}.

    \begin{Verbatim}[commandchars=\\\{\}]
{\color{incolor}In [{\color{incolor}170}]:} \PY{n}{data} \PY{o}{=} \PY{p}{\PYZob{}}\PY{l+s+s1}{\PYZsq{}}\PY{l+s+s1}{Flavor}\PY{l+s+s1}{\PYZsq{}}\PY{p}{:} \PY{p}{[}\PY{l+s+s1}{\PYZsq{}}\PY{l+s+s1}{Strawberry}\PY{l+s+s1}{\PYZsq{}}\PY{p}{,} \PY{l+s+s1}{\PYZsq{}}\PY{l+s+s1}{Vanilla}\PY{l+s+s1}{\PYZsq{}}\PY{p}{,} \PY{l+s+s1}{\PYZsq{}}\PY{l+s+s1}{Chocolate}\PY{l+s+s1}{\PYZsq{}}\PY{p}{]}\PY{p}{,} \PY{l+s+s1}{\PYZsq{}}\PY{l+s+s1}{Price}\PY{l+s+s1}{\PYZsq{}}\PY{p}{:} \PY{p}{[}\PY{l+m+mf}{3.50}\PY{p}{,} \PY{l+m+mf}{3.00}\PY{p}{,} \PY{l+m+mf}{4.25}\PY{p}{]}\PY{p}{\PYZcb{}}
          \PY{n}{icecream} \PY{o}{=} \PY{n}{pd}\PY{o}{.}\PY{n}{DataFrame}\PY{p}{(}\PY{n}{data}\PY{p}{)}
          \PY{n}{icecream}
\end{Verbatim}


\begin{Verbatim}[commandchars=\\\{\}]
{\color{outcolor}Out[{\color{outcolor}170}]:}        Flavor  Price
          0  Strawberry   3.50
          1     Vanilla   3.00
          2   Chocolate   4.25
\end{Verbatim}
            
    Create a bar chart representing the three flavors and their associated
prices.

    \begin{Verbatim}[commandchars=\\\{\}]
{\color{incolor}In [{\color{incolor}173}]:} \PY{n}{plt}\PY{o}{.}\PY{n}{bar}\PY{p}{(}\PY{n}{icecream}\PY{p}{[}\PY{l+s+s1}{\PYZsq{}}\PY{l+s+s1}{Flavor}\PY{l+s+s1}{\PYZsq{}}\PY{p}{]}\PY{p}{,} \PY{n}{icecream}\PY{p}{[}\PY{l+s+s1}{\PYZsq{}}\PY{l+s+s1}{Price}\PY{l+s+s1}{\PYZsq{}}\PY{p}{]}\PY{p}{)}
          \PY{n}{plt}\PY{o}{.}\PY{n}{ylabel}\PY{p}{(}\PY{l+s+s1}{\PYZsq{}}\PY{l+s+s1}{Price}\PY{l+s+s1}{\PYZsq{}}\PY{p}{)}
          \PY{n}{plt}\PY{o}{.}\PY{n}{title}\PY{p}{(}\PY{l+s+s1}{\PYZsq{}}\PY{l+s+s1}{Price of Ice Cream Flavors}\PY{l+s+s1}{\PYZsq{}}\PY{p}{)}
           
          \PY{n}{plt}\PY{o}{.}\PY{n}{show}\PY{p}{(}\PY{p}{)}
\end{Verbatim}


    \begin{center}
    \adjustimage{max size={0.9\linewidth}{0.9\paperheight}}{output_80_0.png}
    \end{center}
    { \hspace*{\fill} \\}
    
    Create 9 random plots. The top three should be scatter plots (one with
green dots, one with purple crosses, and one with blue triangles. The
middle three graphs should be a line graph, a horizontal bar chart, and
a histogram. The bottom three graphs should be trignometric functions
(one sin, one cosine, one tangent).

    \begin{Verbatim}[commandchars=\\\{\}]
{\color{incolor}In [{\color{incolor}263}]:} \PY{k+kn}{import} \PY{n+nn}{random}
          
          \PY{n}{data} \PY{o}{=} \PY{p}{\PYZob{}}\PY{l+s+s1}{\PYZsq{}}\PY{l+s+s1}{col1}\PY{l+s+s1}{\PYZsq{}}\PY{p}{:} \PY{n}{np}\PY{o}{.}\PY{n}{arange}\PY{p}{(}\PY{l+m+mi}{0}\PY{p}{,} \PY{l+m+mi}{50}\PY{p}{)}\PY{p}{,} \PY{l+s+s1}{\PYZsq{}}\PY{l+s+s1}{col2}\PY{l+s+s1}{\PYZsq{}}\PY{p}{:} \PY{n}{np}\PY{o}{.}\PY{n}{random}\PY{o}{.}\PY{n}{normal}\PY{p}{(}\PY{l+m+mi}{10}\PY{p}{,} \PY{o}{.}\PY{l+m+mi}{41}\PY{p}{,} \PY{l+m+mi}{50}\PY{p}{)}\PY{p}{,} \PY{l+s+s1}{\PYZsq{}}\PY{l+s+s1}{col3}\PY{l+s+s1}{\PYZsq{}}\PY{p}{:} \PY{n}{np}\PY{o}{.}\PY{n}{random}\PY{o}{.}\PY{n}{beta}\PY{p}{(}\PY{l+m+mi}{2}\PY{p}{,} \PY{l+m+mi}{3}\PY{p}{,} \PY{l+m+mi}{50}\PY{p}{)}\PY{p}{,} \PY{p}{\PYZcb{}}
          \PY{n}{plot\PYZus{}data} \PY{o}{=} \PY{n}{pd}\PY{o}{.}\PY{n}{DataFrame}\PY{p}{(}\PY{n}{data}\PY{p}{)}
          \PY{n}{bar\PYZus{}data} \PY{o}{=} \PY{n}{pd}\PY{o}{.}\PY{n}{DataFrame}\PY{p}{(}\PY{p}{\PYZob{}}\PY{l+s+s1}{\PYZsq{}}\PY{l+s+s1}{Names}\PY{l+s+s1}{\PYZsq{}}\PY{p}{:} \PY{p}{[}\PY{l+s+s1}{\PYZsq{}}\PY{l+s+s1}{one}\PY{l+s+s1}{\PYZsq{}}\PY{p}{,} \PY{l+s+s1}{\PYZsq{}}\PY{l+s+s1}{two}\PY{l+s+s1}{\PYZsq{}}\PY{p}{,} \PY{l+s+s1}{\PYZsq{}}\PY{l+s+s1}{three}\PY{l+s+s1}{\PYZsq{}}\PY{p}{]}\PY{p}{,} \PY{l+s+s1}{\PYZsq{}}\PY{l+s+s1}{Values}\PY{l+s+s1}{\PYZsq{}} \PY{p}{:} \PY{p}{[}\PY{l+m+mi}{5}\PY{p}{,} \PY{l+m+mi}{10}\PY{p}{,} \PY{l+m+mi}{4}\PY{p}{]}\PY{p}{\PYZcb{}}\PY{p}{)}
          
          \PY{k}{def} \PY{n+nf}{hide\PYZus{}labels}\PY{p}{(}\PY{p}{)}\PY{p}{:}
              \PY{k}{return} \PY{n}{plt}\PY{o}{.}\PY{n}{xticks}\PY{p}{(}\PY{p}{[}\PY{p}{]}\PY{p}{)}\PY{p}{,} \PY{n}{plt}\PY{o}{.}\PY{n}{yticks}\PY{p}{(}\PY{p}{[}\PY{p}{]}\PY{p}{)}
          
          \PY{c+c1}{\PYZsh{}row 1}
          \PY{n}{plt}\PY{o}{.}\PY{n}{subplot}\PY{p}{(}\PY{l+m+mi}{331}\PY{p}{)}
          \PY{n}{plt}\PY{o}{.}\PY{n}{scatter}\PY{p}{(}\PY{n}{data}\PY{p}{[}\PY{l+s+s1}{\PYZsq{}}\PY{l+s+s1}{col1}\PY{l+s+s1}{\PYZsq{}}\PY{p}{]}\PY{p}{,} \PY{n}{data}\PY{p}{[}\PY{l+s+s1}{\PYZsq{}}\PY{l+s+s1}{col2}\PY{l+s+s1}{\PYZsq{}}\PY{p}{]}\PY{p}{,} \PY{n}{color}\PY{o}{=}\PY{l+s+s1}{\PYZsq{}}\PY{l+s+s1}{green}\PY{l+s+s1}{\PYZsq{}}\PY{p}{,} \PY{p}{)}\PY{p}{;} \PY{n}{hide\PYZus{}labels}\PY{p}{(}\PY{p}{)}
          \PY{n}{plt}\PY{o}{.}\PY{n}{subplot}\PY{p}{(}\PY{l+m+mi}{332}\PY{p}{)}
          \PY{n}{plt}\PY{o}{.}\PY{n}{scatter}\PY{p}{(}\PY{n}{data}\PY{p}{[}\PY{l+s+s1}{\PYZsq{}}\PY{l+s+s1}{col1}\PY{l+s+s1}{\PYZsq{}}\PY{p}{]}\PY{p}{,} \PY{n}{data}\PY{p}{[}\PY{l+s+s1}{\PYZsq{}}\PY{l+s+s1}{col3}\PY{l+s+s1}{\PYZsq{}}\PY{p}{]}\PY{p}{,} \PY{n}{color}\PY{o}{=}\PY{l+s+s1}{\PYZsq{}}\PY{l+s+s1}{purple}\PY{l+s+s1}{\PYZsq{}}\PY{p}{,} \PY{n}{marker}\PY{o}{=}\PY{l+s+s1}{\PYZsq{}}\PY{l+s+s1}{x}\PY{l+s+s1}{\PYZsq{}}\PY{p}{)}\PY{p}{;} \PY{n}{hide\PYZus{}labels}\PY{p}{(}\PY{p}{)}
          \PY{n}{plt}\PY{o}{.}\PY{n}{subplot}\PY{p}{(}\PY{l+m+mi}{333}\PY{p}{)}
          \PY{n}{plt}\PY{o}{.}\PY{n}{scatter}\PY{p}{(}\PY{n}{data}\PY{p}{[}\PY{l+s+s1}{\PYZsq{}}\PY{l+s+s1}{col2}\PY{l+s+s1}{\PYZsq{}}\PY{p}{]}\PY{p}{,} \PY{n}{data}\PY{p}{[}\PY{l+s+s1}{\PYZsq{}}\PY{l+s+s1}{col3}\PY{l+s+s1}{\PYZsq{}}\PY{p}{]}\PY{p}{,} \PY{n}{color}\PY{o}{=}\PY{l+s+s1}{\PYZsq{}}\PY{l+s+s1}{blue}\PY{l+s+s1}{\PYZsq{}}\PY{p}{,} \PY{n}{marker}\PY{o}{=}\PY{l+s+s1}{\PYZsq{}}\PY{l+s+s1}{\PYZca{}}\PY{l+s+s1}{\PYZsq{}}\PY{p}{)}\PY{p}{;} \PY{n}{hide\PYZus{}labels}\PY{p}{(}\PY{p}{)}
          
          \PY{c+c1}{\PYZsh{}row 2}
          \PY{n}{plt}\PY{o}{.}\PY{n}{subplot}\PY{p}{(}\PY{l+m+mi}{334}\PY{p}{)}
          \PY{n}{plt}\PY{o}{.}\PY{n}{plot}\PY{p}{(}\PY{n}{data}\PY{p}{[}\PY{l+s+s1}{\PYZsq{}}\PY{l+s+s1}{col1}\PY{l+s+s1}{\PYZsq{}}\PY{p}{]}\PY{p}{,} \PY{n}{data}\PY{p}{[}\PY{l+s+s1}{\PYZsq{}}\PY{l+s+s1}{col2}\PY{l+s+s1}{\PYZsq{}}\PY{p}{]}\PY{p}{)}\PY{p}{;} \PY{n}{hide\PYZus{}labels}\PY{p}{(}\PY{p}{)}
          \PY{n}{plt}\PY{o}{.}\PY{n}{subplot}\PY{p}{(}\PY{l+m+mi}{335}\PY{p}{)}
          \PY{n}{plt}\PY{o}{.}\PY{n}{barh}\PY{p}{(}\PY{n}{np}\PY{o}{.}\PY{n}{arange}\PY{p}{(}\PY{l+m+mi}{0}\PY{p}{,}\PY{l+m+mi}{10}\PY{p}{)}\PY{p}{,} \PY{n}{random}\PY{o}{.}\PY{n}{sample}\PY{p}{(}\PY{n+nb}{range}\PY{p}{(}\PY{l+m+mi}{100}\PY{p}{)}\PY{p}{,} \PY{l+m+mi}{10}\PY{p}{)}\PY{p}{)}\PY{p}{;} \PY{n}{hide\PYZus{}labels}\PY{p}{(}\PY{p}{)}
          \PY{n}{plt}\PY{o}{.}\PY{n}{subplot}\PY{p}{(}\PY{l+m+mi}{336}\PY{p}{)}
          \PY{n}{plt}\PY{o}{.}\PY{n}{hist}\PY{p}{(}\PY{n}{data}\PY{p}{[}\PY{l+s+s1}{\PYZsq{}}\PY{l+s+s1}{col2}\PY{l+s+s1}{\PYZsq{}}\PY{p}{]}\PY{p}{)}\PY{p}{;} \PY{n}{hide\PYZus{}labels}\PY{p}{(}\PY{p}{)}
          
          \PY{c+c1}{\PYZsh{}row3}
          \PY{n}{plt}\PY{o}{.}\PY{n}{subplot}\PY{p}{(}\PY{l+m+mi}{337}\PY{p}{)}
          \PY{n}{plt}\PY{o}{.}\PY{n}{plot}\PY{p}{(}\PY{n}{data}\PY{p}{[}\PY{l+s+s1}{\PYZsq{}}\PY{l+s+s1}{col1}\PY{l+s+s1}{\PYZsq{}}\PY{p}{]}\PY{p}{,} \PY{n}{np}\PY{o}{.}\PY{n}{sin}\PY{p}{(}\PY{n}{data}\PY{p}{[}\PY{l+s+s1}{\PYZsq{}}\PY{l+s+s1}{col1}\PY{l+s+s1}{\PYZsq{}}\PY{p}{]}\PY{p}{)}\PY{p}{)}\PY{p}{;} \PY{n}{hide\PYZus{}labels}\PY{p}{(}\PY{p}{)}
          \PY{n}{plt}\PY{o}{.}\PY{n}{subplot}\PY{p}{(}\PY{l+m+mi}{338}\PY{p}{)}
          \PY{n}{plt}\PY{o}{.}\PY{n}{plot}\PY{p}{(}\PY{n}{data}\PY{p}{[}\PY{l+s+s1}{\PYZsq{}}\PY{l+s+s1}{col1}\PY{l+s+s1}{\PYZsq{}}\PY{p}{]}\PY{p}{,} \PY{n}{np}\PY{o}{.}\PY{n}{cos}\PY{p}{(}\PY{n}{data}\PY{p}{[}\PY{l+s+s1}{\PYZsq{}}\PY{l+s+s1}{col1}\PY{l+s+s1}{\PYZsq{}}\PY{p}{]}\PY{p}{)}\PY{p}{)}\PY{p}{;} \PY{n}{hide\PYZus{}labels}\PY{p}{(}\PY{p}{)}
          \PY{n}{plt}\PY{o}{.}\PY{n}{subplot}\PY{p}{(}\PY{l+m+mi}{339}\PY{p}{)}
          \PY{n}{plt}\PY{o}{.}\PY{n}{plot}\PY{p}{(}\PY{n}{data}\PY{p}{[}\PY{l+s+s1}{\PYZsq{}}\PY{l+s+s1}{col1}\PY{l+s+s1}{\PYZsq{}}\PY{p}{]}\PY{p}{,} \PY{n}{np}\PY{o}{.}\PY{n}{tan}\PY{p}{(}\PY{n}{data}\PY{p}{[}\PY{l+s+s1}{\PYZsq{}}\PY{l+s+s1}{col1}\PY{l+s+s1}{\PYZsq{}}\PY{p}{]}\PY{p}{)}\PY{p}{)}\PY{p}{;} \PY{n}{hide\PYZus{}labels}\PY{p}{(}\PY{p}{)}
          
          \PY{n}{plt}\PY{o}{.}\PY{n}{show}\PY{p}{(}\PY{p}{)}
\end{Verbatim}


    \begin{center}
    \adjustimage{max size={0.9\linewidth}{0.9\paperheight}}{output_82_0.png}
    \end{center}
    { \hspace*{\fill} \\}
    
    \hypertarget{extra-credit}{%
\subsection{Extra Credit}\label{extra-credit}}

    Run the cell below to read in the data. See:
https://www.quantshare.com/sa-43-10-ways-to-download-historical-stock-quotes-data-for-free

    \begin{Verbatim}[commandchars=\\\{\}]
{\color{incolor}In [{\color{incolor} }]:} \PY{n}{df\PYZus{}google} \PY{o}{=} \PY{n}{pd}\PY{o}{.}\PY{n}{read\PYZus{}csv}\PY{p}{(}\PY{l+s+s1}{\PYZsq{}}\PY{l+s+s1}{https://finance.google.com/finance/historical?output=csv\PYZam{}q=goog}\PY{l+s+s1}{\PYZsq{}}\PY{p}{)}
        \PY{n}{df\PYZus{}apple} \PY{o}{=} \PY{n}{pd}\PY{o}{.}\PY{n}{read\PYZus{}csv}\PY{p}{(}\PY{l+s+s1}{\PYZsq{}}\PY{l+s+s1}{https://finance.google.com/finance/historical?output=csv\PYZam{}q=aapl}\PY{l+s+s1}{\PYZsq{}}\PY{p}{)}
        
        \PY{n}{df\PYZus{}disney} \PY{o}{=} \PY{n}{pd}\PY{o}{.}\PY{n}{read\PYZus{}csv}\PY{p}{(}\PY{l+s+s1}{\PYZsq{}}\PY{l+s+s1}{https://finance.google.com/finance/historical?output=csv\PYZam{}q=dis}\PY{l+s+s1}{\PYZsq{}}\PY{p}{)}
        \PY{n}{df\PYZus{}nike}\PY{o}{=} \PY{n}{pd}\PY{o}{.}\PY{n}{read\PYZus{}csv}\PY{p}{(}\PY{l+s+s1}{\PYZsq{}}\PY{l+s+s1}{https://finance.google.com/finance/historical?output=csv\PYZam{}q=nke}\PY{l+s+s1}{\PYZsq{}}\PY{p}{)}
        
        \PY{n}{df\PYZus{}apple}\PY{o}{.}\PY{n}{head}\PY{p}{(}\PY{p}{)}
\end{Verbatim}


    Show a 3 x 3 correlation matrix for Nike, Apple, and Disney stock prices
for the month of July, 2017.

Hint: Convert \texttt{Date} to a pandas datetime object. Change the
indices of all the dataframes to \texttt{Date}. Use \texttt{Date}
indices to filter rows. Create a new dataframe that stores values of the
\texttt{Close} column from each dataframe. Use the \texttt{Close} column
of each company's stock data to find the correlation using df.corr().

    Show the same correlation matrix but over different time periods. 1. the
last 20 days\\
2. the last 80 days

    Change the code so that it accepts a list of any stock symbols (i.e.
{[}`NKE', `APPL', `DIS', \ldots{} {]}) and creates a correlation matrix
for the past 100 days.


    % Add a bibliography block to the postdoc
    
    
    
    \end{document}
